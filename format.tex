% Formatting specifications for University of York theses
% as specified in Regulation 2.8 'Theses submitted for higher degrees' (based on ISO 4821:1990)
% Contains elements of M. Imran's template for University of Durham, Department of Mathematics

% Load useful packages here; you _must_ include 'fancyhdr'
\usepackage{fancyhdr}

% hyperref must be called BEFORE apacite
\usepackage{hyperref}
\usepackage{acronym}
\usepackage{afterpage}
\usepackage{algorithm}
\usepackage{algorithmic}
\usepackage{amsfonts}
\usepackage{amsmath}
\usepackage{amssymb}

\usepackage[notocbib,nosectionbib]{apacite}
% let the doi package handle stuff
\renewcommand{\doiprefix}{}
\usepackage[UKenglish]{babel}

\usepackage{booktabs}
\usepackage[font=small,
			labelfont=bf,labelsep=period]{caption}

\usepackage[usenames, dvipsnames]{color}
\definecolor{shaded}{gray}{0.9}

\usepackage{doi}
\usepackage{epic}
\usepackage{epsfig}
\usepackage{graphicx}
\usepackage{ifthen}
\usepackage{latexsym}
\usepackage{multicol} 
\usepackage{multirow}
\usepackage[super]{nth}
\usepackage{paralist}
\usepackage{pdfpages}
\usepackage{rotating}
\usepackage{sectsty}
\usepackage{siunitx}
\usepackage{theorem}
\usepackage[flushleft]{threeparttable}
\usepackage{units}
\usepackage{wrapfig}

\usepackage{mdframed}
\usepackage{chngcntr}

\usepackage[capitalise,noabbrev]{cleveref}



% Select fonts here; leave all commented to use LaTeX defaults (Computer Modern family)
% I found Computer Modern family really difficult to read on screen
% Note that some fonts tend to give bigger pagecounts: Times < Palatino < ComputerModern < NewCenturySchoolbook
%\usepackage{fontenc}
%\usepackage{times}	% Times, Helvetica, Courier
%\usepackage{newcent}	% New Century Schoolbook, Avant Garde, Courier
%\usepackage{palatino}	% Palatino, Helevetica, Courier


% fancy header
\pagestyle{fancy}
\renewcommand{\chaptermark}[1]{\markboth{\chaptername\ \thechapter:\ #1}{}}
\renewcommand{\sectionmark}[1]{\markright{\thesection\ #1}}
\lhead[\fancyplain{}{\leftmark}]	{\fancyplain{}{}}
\chead[\fancyplain{}{}]			{\fancyplain{}{}}
\rhead[\fancyplain{}{}]			{\fancyplain{}{\rightmark}}
\lfoot[\fancyplain{}{}]			{\fancyplain{}{}}
\cfoot[\fancyplain{}{\thepage}]		{\fancyplain{}{\thepage}}
\rfoot[\fancyplain{}{}]			{\fancyplain{}{}}

% theorem style
{\theorembodyfont{\rmfamily}\newtheorem{Pro}{{\textbf Proposition}}[section]}
{\theorembodyfont{\rmfamily}\newtheorem{The}{{\textbf Theorem}}[section]}
{\theorembodyfont{\rmfamily}\newtheorem{Def}[The]{{\textbf Definition}}}
{\theorembodyfont{\rmfamily}\newtheorem{Cor}[The]{{\textbf Corollary}}}
{\theorembodyfont{\rmfamily}\newtheorem{Lem}[The]{{\textbf Lemma}}}
{\theorembodyfont{\rmfamily}\newtheorem{Exp}{{\textbf Example}}[section]}
\def\remark{\textbf{Remark}:}
\def\remarks{\textbf{Remarks}:}
\def\bproof{\textbf{Proof}: }
\def\eproof{\hfill$\Box$}

%info box
\newcounter{ib}[chapter]
\counterwithin{ib}{chapter}

\newmdenv[backgroundcolor=shaded,
        linewidth=0,
        skipabove=12,
        innertopmargin=-14,
        frametitlerule=false
        ]{info}

\newenvironment{infobox}[1]
{\stepcounter{ib}%
 \begin{info}[frametitle=Box~\theib~#1]
    \noindent
    \linespacesmall
    \footnotesize
    \begin{multicols}{2}}
{\end{multicols}
    \linespacenormal
    \end{info}}