\chapter{General discussions}
\chaptermark{Discussion}
\label{ch:discussion}
%\setcounter{equation}{0}
% ==========================================================================================================
Contemporary research on patterns of self-generated thoughts, such as those occurring during states of mind wandering, are riddled with contradictions. For example, the content of on-going thought varies from future-orientated planning thoughts that may help refine personal goals \cite{Medea2016} from negative past concerns that can maintain unpleasant affective states \cite{Killingsworth2010}. Likewise, research has highlighted in tasks that demand continuous external attention \cite{McVayJOEP2009,McVay2012} whereas evidence from studies of creativity and problem solving suggest evidence of a benefit \cite{Smeekens2016,Baird2012}. 

The conflicting evidence above is thought to emerge because of the variability in the nature of on-going thought. The heterogeneous on-going thought may be composed of a set of experiences with overlapping features---the so called family resemblance account of mind-wandering \cite{Smallwood2013, Seli2018}. Building on this assumption, the focus of this thesis was to develop an empirical approach to sensitive to the possibly similar notions among the heterogeneous thought patterns. In other words, are there multiple patterns of on-going experience that share common and distinct features can be empirically defined. 
A second assumption of this thesis is that understanding the categories of experience can be informed by examining the intersection between subjective reports objective measures---in this case patterns in resting state functional connectivity and performance on cognitive tests. One method that is sensitive to both the assumed heterogeneity and the need to constrain self-reported data by objective metrics is SCCA \cite{WittenSCCA2009}---a multivariate approach that measures similarities of linear patterns in two domains of data. The overarching aim of the thesis is to provide evidence in support of the family resemblance view of on-going thought and facilitate future research on the granularity of the interaction of the biological and behavioural domains.

% ====================================================================================================================
\section{Empirical findings}
\label{ch:discussion:results}

The current thesis focuses on resolving the heterogeneous features of on-going thought by considering its intersection with measures of neural function---in this case the intrinsic functional connectivity at rest. 
\cref{ch:study1} revealed that mind-wandering is a collection of different on-going thoughts that are derived from the connectivity patterns of DMN.% study 1
\cref{ch:study2} found that the population variance in intelligence is related to different whole-brain neural hierarchies and on-going thoughts.% study 2
\cref{ch:study3} showed that either momentary or longer mental representation is associated with the ability to inhibit prior mental sets and the balance of segregation and integration between the limbic system and the rest of the cortex.% study 3
The following section of this thesis moves on to describe in detail the converging evidence of heterogeneity and integration an segregation of transmodal networks. 

\subsection{Heterogeneity}

In the mind-wandering literature, converging evidence highlights heterogeneity in both the variety of  functional outcome linked to off task thought \cite{SmallwoodFrontiers2013} and in both the definitions that researchers have used to study different type of on-going thought \cite{SeliTiCS2016} and the number of competing theoretical accounts \cite<i.e.>{Smallwood2010,McVay2010}. The current thesis aimed to systematically explore whether using the tools of cognitive neuroscience and experience sampling it is possible to empirically demonstrate the heterogeneity of experience. In particular, \cref{ch:study1} explored whether the heterogeneity of ongoing thought can emerge from a single candidate system—the DMN. \cref{ch:study2} examined whether the heterogeneity in patterns of ongoing thought can be realised from whole-brain analysis.

It is a widely held view that the DMN is important in states of mind-wandering \cite<see review from>{SmallwoodSchooler2015}. Building on this, the aim of \cref{ch:study1} was to explore whether the conjoin patterns of within DMN connectivity and on-going thought predict unique patterns of functional outcomes, and thus account for the variety of functional outcomes linked to ongoing thought. Two neuro-experiential patterns were relate to distinct functional outcomes. Internal connectivity in mPFC related to positive habitual experience was predicted by poor executive control \cite{McVay2009}. The on-going thought related to deficit in executive control could be a result of failure to allocate resource to the external task. The PCC-TPJ-mPFC negative connection chain conjoining with off-task experience revealed internal generation of information. The continuous content generation associated with external task is similar to the association of mind wandering and creativity \cite{Baird2012}. The data provided evidence that on-going thoughts unfold along a set of heterogeneous dimensions and critically, explains how the conflicts between the representational and executive failure accounts could be a consequences of different configurations in the DMN.

Since the brain works as a system when engaging in tasks, a natural question following \cref{ch:study1} is how regions other than DMN contribute to the heterogeneity of ongoing thought. Recent works on hierarchical functional cortical organization at rest suggested variability in relationships among large-scale networks \cite{Margulies2016}. In \cref{ch:study2}, the focus shifted from the DMN to the whole brain in order to explore the contribution of other brain networks. The results suggested that among the discovered neuro-experiential components, different neural hierarchies were involved during on-going thoughts. Purposeful monologue associated with sensory segregation—dissociation between the DMN and unimodal networks— at whole brain level and was predicted by high intelligence. This pattern is well documented in the semantic memory literature, showing that the ability to generate abstract representations relies on the segregation of internal and external representation of the memory \cite{Murphy2018}. These data also suggested that dysfunctions within the MD network \cite{Duncan2010} may contribute to poor intelligence. Individuals whose ongoing thoughts were directed towards their personal concerns, and who also had low levels of internal connectivity within the attention networks, tended to do worse at measures of intelligence and control, suggesting that this phenotypical pattern has the hallmarks of executive failure \cite<i.e.>{McVay2010}. In contrast to data present in \cref{ch:study1}, where a singular system gives rise to different patterns of thoughts and behavioural, the data presented in \cref{ch:study2} shows that heterogeneous patterns of ongoing thought also emerge when the brain is considered as a large scale dynamic system.

\subsection{Integration and segregation in transmodal networks}

A second theme of the thesis is evidence that patterns of heterogeneity rely on different patterns of integration and segregation between neural systems. Understanding the consequence of integration and segregation is thought to be an important principle in brain organisation. For example, hierarchical integration of sensory information is thought to support more abstract aspects of cognition \cite{Mesulam1998}. In contrast, segregating neural systems is thought to provide flexibility in the operations that can be performed \cite{Buckner2013}. A hierarchical organisation implicating both integration and segregation is captured by the primary gradient which stretches from the unimodal to the transmodal networks \cite{Margulies2016}. Other examples of cognitive hierarchies that depend on integration and segregation include the MDN \cite{Duncan2010}. This large scale network is important whenever individuals perform complex goal directed tasks. Neurally, studies have shown that the MDN involves integration between attention and control systems, and segregation of these systems from the DMN. Notably, the principle gradient and the MDN are differentiable in terms of the degree of separation between the DMN and FPN. The limbic system hierarchy has emerged from the current data. The limbic system is connected internally while functionally adjoins to DMN and FPN, separating the transmodal system from the MDN and sensorimotor system. Similar to DMN, the limbic system hierarchy has been found to facilitate and regulate mental representation, while engaging in a variety of cognitive tasks centred around value-based decision making \cite{Kleckner2017}. 
These and other studies illustrate that patterns of integration are important aspects of neural organisation, and several aspects of this thesis results are related to this. 

The analysis in \cref{ch:study1} is related to patterns of integration and segregation because it shows that different types of patterns of ongoing thought can emerge from the integration of neural information into the DMN. One pattern of thought that reflected positive and habitual content and was linked to poor executive control, was linked to stronger coupling with a number of regions outside of the DMN including left temporoparietal cortex, left hippocampus/entorhinal cortex, left lateral middle temporal gyrus, and the left pre-SMA, also coupled with DMN to support memory representation. 

We also found evidence of integration and segregation in \cref{ch:study2}. One pattern of ongoing thought that was evolving around future planning . This pattern of thought was linked to segregation between DMN and the primary sensory systems and was associated with good performance on intelligence tests tasks. This provides evidence of the importance of network segregation for effective cognition. In contrast, a second pattern of thoughts reflecting personal importance . In neural terms this pattern was linked to reduced connectivity (i.e. lower integration) within many regions of attention and control systems. Together this pattern suggests that a failure to integrate neural functioning within the MDN can give rise to problems in cognition. Thus in \cref{ch:study2} we found evidence of both the importance of integration and segregation. 

Finally, \cref{ch:study3} highlighted the integration and segregation in the limbic system. The limbic system has been found to form a hierarchy facilitating attention inhibition behaviours \cite{Kleckner2017}. In this study we found a pattern of neural organisation that involved highly inter-connected limbic system integrating with the other transmodal area and segregated from the sensorimotor system and that was predictive of behaviour that entailed flexibility of retaining mental content. Importantly, we found that this pattern of population variation was linked to patterns of ongoing thought that varied from task-focused thought at one end to personal, habitual content at the other. This analysis suggests that the degree of integration and segregation between the limbic system and other areas of cortex is a primary determinant of patterns of ongoing thought.


\section{Limitations and future directions}

There are a number of important limitations that should be borne in mind. The current thesis focused on individual differences in a population rather than state level of on-going thought. The data sets used in the current analysis focused on resting-state fMRI and individual-level feature on cognitive functions. The benefit of resting state data set is the wide variety of task selections due to the convenience in data collection. The discovery presented in the current thesis is therefore limited to individual level. The two limitations of the thesis are both possibly related to the focus on population variations. 

Firstly, the current research under-utilise the dynamic nature of resting state by extracting a individual level connectivity profile. Resting state fMRI data has been found to display both trait and state like features \cite{Geerligs2015}. On the other hand, non-biophysical model of the resting state showed changes in the functional connectivity in the resting state \cite{Vidaurre2017}. 

The second limitation is the unclear structure among the component processes. On-going thought is a dynamic change of states \cite{KucyiNI2017}. Researches on state-related features of on-going thought may provide more insight on the interaction of different component processes.
 
Before concluding, it is also worth considering the implications of the results of the thesis for future studies. The following section will discuss these possible future research directions, including how to explore dynamic changes in network connectivity and the potential to gain insights on architecture of the component processes.

\subsection{Shift to states rather than traits}

The current thesis examined the variation of on-going thought in a population with the trait-level behavioural measures. Individual traits has been explore in the past and provided promising results related to on-going thought \cite{Smallwood2016,McVay2009,RubyPlos2013} and intrinsic neural organisation \cite{Smith2015}. The degree of overlap between the results of studies examining population variation in ongoing thought, with studies that explore their momentary features is currently unknown. To address this issue, future work could explore the relationship between measures of neural dynamics and patterns of ongoing thought \cite{KucyiNI2017}. Sliding window analysis \cite{Chang2010} or hidden Markov models \cite{Vidaurre2017} on resting state data could provide complementary information to the individual level observations, such as the structure of trait-level component processes or state specific patterns would not be captured in the current analysis.

The change of state in on-going thought can be explored in task experiments with neuroimaging measures. The ideal task design would be collecting rich behavioural profiles yielding multivariate responses. A recent example is the fMRI version of the 0-back/1-back experience sampling task (see \cref{study1:method:d} in \cref{ch:study1}) used by \citeA{Sormaz2018} to explore the activation of DMN in tasks. The design of this fMRI version prolong the post-probe response time in order to prevent the task from under-/over-run in the scanner. This compensation results in less experience probes being collected. To improve the current design while matching the task and scanner run time, the number of presented trials should be constantly re-evaluated and generate suitable number of trials. The improved version will fit the scanner run-time and be more similar to the laboratory version. The current n-back task does not synchronise the presentation on screen and the on-set of the TR. A improved design would coordinate the presentation on screen with the on-set of TR, thus the on-screen information. 

Neuro-cognitive components have provided promising insight to the component processes under the family resemblance framework of on-going thought. Application of CCA to yield neuro-cognitive pattern at state level can potentially provide fruitful discovery. The challenge of CCA applications on task data will not be in the CCA model selection itself, but in searching for the suitable features for evaluation. The task will need to collect sufficient rich neural and behavioural information at a short time window. MDES accesses multiple aspects of experience and provides desirable multivariate data for CCA. The calculation of functional connectivity is more controversial. I continue to discuss the main challenges of functional connectivity in the next section.

\subsection{Whole brain functional connectivity as a convert marker}

This thesis provided evidence of the neuro-cognitive component processes of on-going thought derive its heterogeneity at individual level. The variation in whole-brain functional hierarchy potentially supports different types of on-going thought. \cref{ch:study2} demonstrated structurally dissimilar neuro-experiential components can be associated with similar cognitive functions\cite{Buckner2013}. Large-scale networks discovered at group level showed resemblance to network organisation at individual level \cite{Finn2015}. Task potency comparing the difference of functional connectivity between task and rest has been proven to predict age difference in context switching related tasks \cite{Chauvin2018}. 

Functional neural hierarchy was the crucial element for uncovering the family resemblance behind the heterogeneity of on-going thought. Study in the future should adopt whole-brain functional connectivity measures for on-going thought. To capture the neural hierarchy, mapping multiple brain regions is necessary. Functional connectivity is crucial to elaborate the organisations among the selected regions. Metrics such as ROI-based functional connectivity, or connectivity-driven gradients \cite<e.g. diffusion embedded gradients>{Margulies2016,Marquand2017} are efficient for summarising interactions among regions across time. 

Functional connectivity at rest have provided promising results in the current thesis, however the need of time-series property for the calculation constrain the application on on-line neuroimaging task. The temporal resolution of fMRI might not provide enough sample to capture the timely state dynamic.If we sample the on-going thought in a task, there needs to be sufficient time-series for calculating functional connectivity to present the neural activity.  Magnetoencephalography (MEG) might be more suitable to gain statistical power for the functional connectivity calculation. Another standing question is the balance between sufficient time-series for functional connectivity calculations and the sensible time window capturing the on-going thought. The emergence of a state is a continuous, gradual process with no obvious starting point. Introducing time-window in the analysis will evoke discussion on the definition of the starting point and the length of a state. To implement the neural hierarchy measures with experience sampling, the research will need to find a pragmatic solution to these potential issues. 

\subsection{SCCA}

The three studies in the current thesis explored and improved the model selection strategy of SCCA. The analysis in \cref{ch:study1} did not select the hyper-parameters in a data-driven manner. With formal hyper-parameter selection, \cref{ch:study2} is more transparent and data-driven in the model selection. A nested cross-validation scheme was adopted to simultaneously select the hyper-parameter and the final model. With the motivation to construct a pipeline that can be generalised to the basic version of linear CCA, \cref{ch:study3} separate the hyper-parameter selection step and the mode selection. The final canonical correlations of the principle mode improved from 0.28 in \cref{ch:study2} to 0.70 in \cref{ch:study3} with a simpler pipeline. The scope of the current thesis focuses on the psychological question of on-going thought, hence the two pipelines presented in \cref{ch:study2,ch:study3} are not formally bench-marked on the same data set. Future work a structurally simple and well-performed pipeline would be important for the application of CCA and its variation on biomedical data.

The other concern is the choice of optimisation target for model selection. The current thesis uses out-of-sample explained variance as the learning target. The rationale is to maximise the potential of predictability in a wider unknown sample with the limited sample size. The alternative choice would be the out-of-sample prediction error, which minimise the mistake when applied to a unknown sample. This thesis did not explore the second option, hence the performance is unknown. These two optimisation targets are asking two fundamentally different questions---explained variance provides a more optimistic view of the model, while prediction error is more conservative. It is uncertain whether the choice of learning target should be question-driven or performance-driven. Again, a bench-marking study would be helpful to clarify the potential of the options. 



% ==========================================================================================================

\section{Concluding remarks}
\label{ch:discussion:summary}

This thesis set out to examine the neuro-cognitive mechanism of on-going thought and establish the basic component processes to incorporate the heterogeneity of on-going thought. Three major questions were posed at the start of the thesis. These will now be revisited in light of the work performed.

% The heterogeneous functional outcome of different type of on-going thought is well documented in the literature. However, the underlying mechanism of on-going thought and the reason of its heterogeneity remains a mystery. On-going thought during unconstrained state is associated with the emergence of DMN. Recent discovery of neural hierarchy in unconstrained processes reveals a network can be involve in multiple configuration. DMN emerges demanding task and internal content representation each with their unique whole brain configuration. These evidence suggested unconstrained processes such as on-going thought would be heterogeneous with the the wide variation of neural hierarchical organisation. The current thesis aims to uncover the neuro-cognitive mechanism of on-going thought and establish the basic component processes to incorporate the heterogeneity of on-going thought. The evidences from the current thesis suggests on-going thought is a family of heterogeneous member with overlapping similarities. Distinct whole brain neural hierarchies support different functional associations of self-generated on-going thought. DMN and the limbic system were found to interact with the whole brain in different manner while both derived from internally generated thought. The current thesis provided individual differences evidence suggesting that the on-going thought is a family of component processes that generates heterogeneous profiles. However, the structure of the component processes remains unknown. Future studies on on-going thought or unconstrained process should consider a multivariate whole-brain approach to consider the hierarchy in cortical organisation rather than individual regions. Investigation of transient on-going thought will be the future direction of research for a deeper understanding of the structure behind the component processes. 

\textbf{Why does on-going thought show both costs and benefits?} On-going thought is a collective phenomena with multiple types of spontaneous thoughts and each type has their own associated functional outcomes at the trait level. The conflicting theoretical accounts of mind wandering should be treated as complementary component processes of on-going thought rather than isolated types of on-going experience. Further work on the structure of component processes will be important to understand the emergence of different experiences through the balance among different processes. 

\textbf{Can functional neural hierarchy explain the heterogeneity?} On-going thoughts with similar experimental profiles or functional outcomes can be driven by different neural hierarchy revealed via whole brain functional connectivity. Each transmodal large-scale network has the potential to be the centre of global integration or segregation of other cortical regions. Further work is suggested to incorporate the neural basis with the on-going thought profiles at the state level to understand the dynamic of on-going thought. 

\textbf{Is the family resemblance view viable for on-going thought?} The on-going thought is potentially a family of component processes that generates heterogeneous profiles by common neural hierarchies with different whole-brain integration or segregation patterns and shared abstract level cognitive functions such as memory representation and executive controls. The current thesis found common component processes that derived population variation. Further work is necessary on the application of these findings at a state level within individuals and the architecture of the component processes. 

In conclusion, the overall thesis provide the proof of principle of the family resemblance view of on-going thought at individual level. The different functional organisation patterns of transmodal network is the potential key to integrate the heterogeneous spontaneous thought under the same framework. Further study should incorporate functional connectivity to understand on-going thought. All the results presented in the thesis are based on results at individual level. Applications of these results on interpreting state level dynamics should be proceed with cautious. 