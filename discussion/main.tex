\chapter{General discussions}
\chaptermark{Discussion}
\label{ch:discussion}
%\setcounter{equation}{0}
% ==========================================================================================================
Contemporary research on patterns of self-generated thoughts, such as those occurring during states of mind wandering, are riddled with contradictions. For example, the content of on-going thought varies from future-orientated planning thoughts that may help refine personal goals \cite{Medea2016} from negative past concerns that can maintain unpleasant affective states \cite{Killingsworth2010}. Likewise, research has highlighted in tasks that demand continuous external attention \cite{McVayJOEP2009,McVay2012} whereas evidence from studies of creativity and problem solving suggest evidence of a benefit \cite{Smeekens2016,Baird2012}.

The conflicting evidence above is thought to emerge because of the variability in the nature of on-going thought. On-going thought are heterogeneous, composed of experiences with overlapping features—the so called family resemblance account of mind-wandering \cite{Smallwood2013, Seli2018}. One aim of this thesis was to develop an empirical approach sensitive to the possibly of heterogeneity within patterns of thought. In particular we aimed to identify multiple patterns of on-going experience that share common and distinct features that can be empirically measured. In order to implement this goal this thesis examined the intersection between subjective reports and objective measures---in this case patterns in resting state functional connectivity and performance on cognitive tests. We employed Sparse Canonical Correlation Analysis \cite{WittenSCCA2009} ---a multivariate approach that measures similarities of linear patterns in two domains of data. This allowed us to generate dimensions that reflect the covariance between both brain and cognition. Using this approach we hoped to provide evidence in support of the family resemblance view of on-going thought that would facilitate more constrained theoretical accounts of how different patterns of ongoing thought emerge.

% ====================================================================================================================
\section{Empirical findings}
\label{ch:discussion:results}

The current thesis comprised three studies focused on resolving the heterogeneous features of ongoing thought by considering its intersection with measures of neural function---in this case the intrinsic functional connectivity at rest. 
\cref{ch:study1} revealed that mind-wandering is a collection of different ongoing thoughts that are derived from the connectivity patterns of DMN.% study 1
\cref{ch:study2} found that the population variance in intelligence is related to different whole-brain neural hierarchies and ongoing thoughts.% study 2
\cref{ch:study3} showed that either momentary or longer mental representation is associated with the ability to inhibit prior mental sets and the balance of segregation and integration between the limbic system and the rest of the cortex.% study 3
This section of this thesis describes in detail two themes that emerged from this work---the heterogeneity of patterns of ongoing thought and the integration and segregation of transmodal networks.

\subsection{Heterogeneity}

In the mind-wandering literature, converging evidence highlights heterogeneity in both the variety of  functional outcome linked to off task thought \cite{SmallwoodFrontiers2013} and in the definitions that researchers have used to study different type of ongoing thought \cite{SeliTiCS2016} and the number of competing theoretical accounts \cite<e.g.>{Smallwood2010,McVay2010}. The current thesis aimed to systematically explore whether this conflicted literature emerges due to the emergence of distinct patterns of ongoing thought with different experiential features and associated outcomes. We explored this question by focusing on a single candidate neural system—the DMN (\cref{ch:study1}) and at the whole brain level (\cref{ch:study2}).

It is a widely held view that the DMN is important in certain types of ongoing thought \cite<see review from>{SmallwoodSchooler2015}. The aim of Study 1 was to explore whether associations between patterns of within DMN connectivity and measures of experience yield unique patterns with distinctive patterns of functional outcomes. In this study, we found two reliable neuro-experiential patterns each with distinct functional outcomes. Internal connectivity in mPFC related to positive habitual experience was predicted by poor executive control, suggesting that this may correspond to patterns of executive failure linked to the mind-wandering state \cite{McVay2009}. The ongoing thought related to deficit in executive control could be a result of failure to allocate resource to the external task. In addition, patterns of PCC-TPJ-mPFC decoupling, associated with off-task experience, was associated with better performance on tasks requiring the generation of information.  The continuous content generation associated with external task is similar to the association between patterns of off-task thouhgt and creativity \cite{Baird2012}. Together, \cref{ch:study1} provided evidence that ongoing thoughts unfold along a set of heterogeneous dimensions and critically, explains how the conflicts between the representational and executive failure accounts could be a consequences of different configurations in the DMN.

Since the brain works as a distributed system when engaging in tasks, a natural question following \cref{ch:study1} is how regions other than DMN contribute to the heterogeneity of ongoing thought. Recent works on hierarchical functional cortical organisation at rest suggested variability in relationships among large-scale networks \cite<e.g.>{Margulies2016}. In \cref{ch:study2}, the focus shifted from the DMN to the whole brain in order to explore the contribution of other brain networks to the heterogeneity of ongoing experience. The results suggested that the neuroexperiential components identified by these analyses suggest that different patterns of ongoing thought may be linked to distinctive neural hierarchies. For example, experiences characterized as purposeful monologue where linked to sensory segregation—dissociation between the DMN and unimodal networks— at whole brain level. This pattern of decoupling is a well documented element of ongoing thought \cite<see>{Smallwood2013,SmallwoodFrontiers2013} and is thought to reduce the interference between patterns of self-generated thought and events in the external environment. Consistent with the view that this process is adaptive, this patterns of experience was linked to better performance on measures of cognition and intelligence. Study also highlighted that dysfunctions within a second hierarchy – the multi demand network \cite{Duncan2010} may also make an important contribution to patterns of ongoing thought. Individuals whose thoughts were directed towards their personal concerns, had low levels of connectivity within both the venral and dorsal attention networks, as well as the fronto parietal network. Critically, these individuals tended to do worse at measures of intelligence and control, suggesting that this phenotypical pattern has the hallmarks of executive failure \cite<i.e.>{McVay2010}. In contrast to data present in \cref{ch:study1}, where a singular system gives rise to different patterns of thoughts and behavioural, the data presented in \cref{ch:study2} shows that multiple heterogeneous patterns of ongoing thought emerge from patterns of connectivity that reflect previously documented neuro cognitive hierarchies.

The demonstration that multiple overlapping patterns of ongoing thought can be realized by combining experiential data with objective neuro-cognitive measures suggests that some of the theoretical controversies related to ongoing thought can be explained as relating to distinct patterns of ongoing thought. For example, in both Study 1 and 2 we found certain patterns of ongoing experience that have beneficial features (such as better creativity or intelligence) and others with less beneficial correlates (such as lower intelligence or worse executive control). Based on these findings it seems likely that at least some of the controversies regarding whether mind-wandering should considered a failure of executive control \cite<e.g.>{McVay2010,Smallwood2010} results from prior studies lumping together of experiential states with different features into a single category. In other words one important contribution of this thesis is that because because ongoing thought contains categories of experience then different theoretical positions that were initially seen as competing elements of the same phenomena, can be be seen to reflect independent aspects of ongoing thought. Our capacity to identify these overlapping patterns of experience is made possible in part because of the use of a biological marker (i.e. functional connectivity) as well as assessments on multiple aspects of ongoing thought and task performance. Moving forward, studies of ongoing thought would therefore benefit from measuring multiple dimensions of experience, as well as measures of covert function such as neuroimaging measures, pupillometry\cite{Konishi2015}, or other biological measures. 


\subsection{Integration and segregation in transmodal networks}

A second theme emerging from this thesis is evidence that patterns of heterogeneity emerge through differential patterns of integration and segregation between and within large scale neural systems. Integration and segregation are both assumed to be an important principle in brain organisation. 
%DMN
For example, hierarchical integration of sensory information is thought to support more abstract aspects of cognition \cite{Mesulam1998}. In contrast, segregating neural systems is thought to provide flexibility in the operations that can be performed \cite{Buckner2013}. A hierarchical organisation implicating both integration and segregation is captured by the primary gradient which stretches from the unimodal to the transmodal networks \cite{Margulies2016}.
%MDN
Other examples of cognitive hierarchies that depend on integration and segregation include the MDN \cite{Duncan2010}. TThe integrated activity of large scale network concerned with integration and segregation are important whenever individuals perform complex goal directed tasks.  Notably, the principle gradient and the MDN are differentiable in terms of the degree of separation between the DMN and FPN, indicating that a defining features of what makes these neurocognitive hierarchies different is how patterns of integration take place within transmodal cortex. 
%limbic (need work)
Other hierarchies are thought to emerge from the limbic system which is thought to describe a hierarchy composed of visceromotor regions that connect with DMN, and salience network \cite{Kleckner2017}. These authors suggest that this forms an allostatic-interoceptive system, segregated from the unimodal and attention systems. This hierarchy is assumed to emerge because the limbic system can selectively integrate information from systems involved in attention and cognition, as well as those important for emotion and affect. 
%short summary on past study
These past studies illustrate that at the core of different neural hierarchies are patterns of integration and segregations between distributed neural regions.

One important implication of this thesis is that it underscores the importance of integration and segregation in the neural hierarchies that support patterns of ongoing thought. \cref{ch:study1} demonstrated that the role of the DMN in distinct patterns of ongoing thought emerges because of differences in the integration and segregation within this system, with spontaneous off task thought linked to lower connectivity within this system, while positive habitual thought was linked to integration within the same system. Importantly, this latter pattern of experience was also linked to stronger coupling with a number of regions outside of the DMN including left temporoparietal cortex, left hippocampus/entorhinal cortex, left lateral middle temporal gyrus, and the left pre-SMA. We also found evidence of integration and segregation in \cref{ch:study2}. One pattern of ongoing thought that was linked to purposeful future planning and was linked by segregation between DMN and the primary sensory systems. A second pattern of thoughts reflected ongoing thoughts of personal importance and this was linked to reduced connectivity (i.e. lower integration) within many regions of attention and control systems. Thus within both \cref{ch:study1,ch:study2} patterns of ongoing thought were differentiable based on the patterns of integration and segregation between neural systems.

The most powerful evidence of integration and segregation within this thesis is provided by \cref{ch:study3}. This analysis highlighted the integration and segregation in the limbic system as a core determinant of patterns of ongoing thought. The limbic system has been previously argued to form a hierarchy integrating the transmodal system while simultaneously segregating the unimodal system to facilitate attention inhibition \cite{Kleckner2017}. In Study we found a pattern of population variation anchored at one end by a highly inter-connected limbic system integrating with the other transmodal area and segregated from the sensorimotor system and that was predictive of behaviour that entailed flexibility of retaining mental content. At the other extreme the limbic system was highly coupled with neural system and individuals were unable to inhibit their prior mental set. Importantly, we found that this pattern of population variation was linked to patterns of ongoing thought that varied from task-focused thought at one end to personal, habitual content at the other. This analysis not only suggests that the degree of integration and segregation between the limbic system is important for attentional control \cite<i.e.>{Kleckner2017} it also suggests that the degree to which this system is coupled or decouple from other aspects of the cortex is a primary determinant of patterns of ongoing thought are focused on the task, or are instead focused on personally relevant matters in detailed manner.

Together the three studies presented in this thesis show that at the core of different patterns of ongoing thought are the integration and segregation between neural systems. Moving forward studies should formally consider how patterns of integration and segregation between neural systems can give rise to the heterogeneity of patterns of ongoing thought that make up our daily lives.


\section{Limitations}

% trait vs state
The primary limitations of this work is how it dealt with the temporal elements of cognition. For example, the studies focused exclusively on individual differences within a population rather than state level of ongoing thought. Studies exploring the associations between static functional connectivity and psychological traits have brought fruitful results to ongoing thought \cite{Smallwood2016,McVay2009,RubyPlos2013}. However, it is import to bear in mind that these studies confound traits with states, since a defining feature of patterns of ongoing thought is their intermittent nature. While individual traits allow a way to understand links between cognition and the brain, it remains to be seen whether the patterns discovered at the population level will be applicable in momentary state needs more exploration. 

There are two ways that future studies could provide a more valid temporal perspective on patterns of ongoing thought. One approach would be to measure experience and neural activity on multiple occasions. One potential strategy is collecting multiple examples of experience using online probes while neural function is recorded. Recently using the same 0-back/1-back task in conjunction with measures of neural function provided by fMRI we found that different dimensions of experience can have unique neural representations \cite{Sormaz2018}. It would be possible to apply CCA to data collected in this manner which would allow neurocognitive patterns to emerge that describe momentary states rather than population variation. A second approach would be to explore the association between patterns of dynamic neural function and ongoing experience. Recent discovery of temporal dynamic using hidden Markov models \cite<HMM;>{Vidaurre2017} demonstrated that time spent in brain states predicts behavioural traits including measures of inhibition control and attention. HMM allow the identification of temporally re-ocurring states that are defined by similarity between neural data across time. It is possible that the application of HMM to neural data would resolve covert patterns of neural function that reflect momentary changes in experience that reflect different patterns of ongoing thought.

Another limitation emerges from the statistical technique that was applied in this thesis in terms of both model selection strategy employed. The three studies in the current thesis explored and improved the model selection strategy of SCCA. The analysis in \cref{ch:study1} did not select the hyper-parameters in a data-driven manner. With formal hyper-parameter selection, \cref{ch:study2} is more transparent and data-driven in the model selection. A nested cross-validation scheme was adopted to simultaneously select the hyper-parameter and the final model. With the motivation to construct a pipeline that can be generalised to the basic version of linear CCA, \cref{ch:study3} separate the hyper-parameter selection step and the mode selection. The final canonical correlations of the principle mode improved from 0.28 in \cref{ch:study2} to 0.70 in \cref{ch:study3} with a simpler pipeline. The scope of the current thesis focuses on the psychological question of ongoing thought, hence the two pipelines presented in \cref{ch:study2,ch:study3} are not formally bench-marked on the same data set. Future work a structurally simple and well-performed pipeline would be important for the application of CCA and its variation on biomedical data.

The other concern is the choice of optimisation target for model selection. The current thesis uses out-of-sample explained variance as the learning target. The rationale is to maximise the potential of predictability in a wider unknown sample with the limited sample size. The alternative choice would be the out-of-sample prediction error, which minimise the mistake when applied to a unknown sample. This thesis did not explore the second option, hence the performance is unknown. These two optimisation targets are asking two fundamentally different questions---explained variance provides a more optimistic view of the model, while prediction error is more conservative. It is uncertain whether the choice of learning target should be question-driven or performance-driven. Again, a bench-marking study would be helpful to clarify the potential of the options. 



\section{Future directions}

Before concluding it is worth considering the implications for two specific areas of the study of ongoing thought. Much debate has been around the intermittent disruption caused by experience sampling methods when intending to measure the train of thought in a concurrent task \cite{SmallwoodSchooler2006}. This problem is especially concerning with MDES, where participants spend around a minute to report the thought rather than one or two questions that can be done in seconds. A covert marker would allow to study the ongoing thought while not interrupting the natural flow of thought.

This thesis showed that the variation in whole-brain functional hierarchy potentially supports different types of ongoing thought. If, as this thesis suggests, patterns of integration and segregation in neural activity are important aspects of different features of ongoing thought, then the covert marker could be based on patterns of functional connectivity . However, the calculation of connectivity depends on the processing of time-series data making the determination of rapid temporal changes problematic. This is compounded by the low temporal resolutions It is possible that it may be important in this context to use MEG for the determination of online marker since the ability of this approach to reveal neural processes at the level of milliseconds superior to fMRI.

In conclusion, the current thesis provided a proof of principle on the utility of whole-brain functional connectivity in exploring ongoing thought. It has the potential to be the covert online marker for spontaneous thought. However, with the current limitation in fMRI temporal resolution, functional connectivity calculation would be the main challenge of such application. MEG is the possible candidate method for understanding the dynamic of ongoing thought and underlying neural pattern.


% ==========================================================================================================

\section{Concluding remarks}
\label{ch:discussion:summary}

This thesis set out to examine the neuro-cognitive mechanism of ongoing thought and establish the basic component processes to incorporate the heterogeneity of ongoing thought. Three major questions were posed at the start of the thesis. These will now be revisited in light of the work performed.

\textbf{Why does ongoing thought show both costs and benefits?} Ongoing thought is a collective phenomena with multiple types of experience each with their own associated functional outcomes at the trait level. This thesis suggests that pattern of costs and benefits related to mind-wandering may be usefully conceptualised as characterising overlapping but distinct aspects of ongoing experience. Further work will be important to understand the underlying mechanisms that explain why these different states emerge.

\textbf{Can functional neural hierarchy explain the heterogeneity?} This thesis demonstrated that ongoing thoughts with different experimental profiles are associated with different neural hierarchy. Further work is suggested to incorporate the neural basis with the ongoing thought profiles at the state level to understand the dynamic of ongoing thought.

\textbf{Is the family resemblance view viable for ongoing thought?} Overall this thesis supports the contention that ongoing thought can be conceived of as a family of experiences with similar and overlapping features. The current thesis found common component processes that determine population variation and further work is necessary on the application of these findings at a state level within individuals and the architecture of the component processes.