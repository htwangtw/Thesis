\chapter{General discussion}
\chaptermark{Discussion}
\label{ch:discussion}
%\setcounter{equation}{0}
% ==========================================================================================================
Contemporary research on patterns of self-generated thought, such as those occurring during states of mind-wandering, is riddled with contradictions. The content of ongoing thought varies from future-orientated planning thoughts that may help refine personal goals \cite{Medea2016} to negative past concerns that can maintain unpleasant affective states \cite{Killingsworth2010}. Likewise, research has highlighted the disadvantage of off-task thought during tasks that demand continuous external attention \cite{McVayJOEP2009,McVay2012}, whereas research on creativity and problem solving suggests evidence of beneficial influence from off-task thought\cite{Smeekens2016,Baird2012}.

The conflict presented above is thought to emerge because of variability in the nature of ongoing thought. Heterogeneous ongoing thought may be composed of a set of experiences with overlapping features---the so-called family resemblance account of mind-wandering \cite{Smallwood2013, Seli2018}. One aim of this thesis was to develop an empirical approach sensitive to possible similarities among the heterogeneous patterns of thought.
In particular, the goal was to identify multiple patterns of ongoing experience that share common and distinct features that can be empirically measured. To implement this goal, this thesis examined the intersection between subjective reports and objective measures---in this case patterns of resting-state functional connectivity and performance on cognitive tests. The current thesis employed SCCA \cite{WittenSCCA2009}---a multivariate approach that measures similarities of linear patterns in two domains of data. SCCA identifies various patterns of thought while reflecting the covariance between both brain and cognition.
The overarching aim of this thesis is to provide evidence in support of the family resemblance view of ongoing thought that, and facilitate a more constrained theoretical account of how different patterns of ongoing thought emerge.

% ====================================================================================================================
\section{Empirical findings}
\label{ch:discussion:results}

The current thesis comprised three studies focusing on resolving the heterogeneous features of ongoing thought by considering its intersection with measures of neural function---in this case intrinsic functional connectivity at rest.
\cref{ch:study1} revealed that mind-wandering is a collection of different ongoing thoughts that are derived from the connectivity patterns in DMN. % study 1
\cref{ch:study2} found that the population variance in intelligence is related to different whole-brain neural hierarchies and ongoing thoughts. % study 2
\cref{ch:study3} showed that either momentary or longer mental representation is associated with the ability to inhibit prior mental sets and the balance of segregation and integration between the limbic system and the rest of the cortex. % study 3
describes in detail two themes that emerged from this work---the heterogeneity of patterns of ongoing thought and the integration and segregation of transmodal networks.

\subsection{Heterogeneity}

In mind-wandering literature, converging evidence highlights heterogeneity in the variety of functional outcome linked to off-task thought \cite{SmallwoodFrontiers2013}, the definitions that researchers have used to study different types of ongoing thought \cite{SeliTiCS2016}, and the number of competing theoretical accounts \cite<e.g.>{Smallwood2010,McVay2010}. The current thesis aimed to systematically explore whether this conflicted literature is a consequence of the emergence of distinct patterns of ongoing thought with different experiential features and associated outcomes. This question was explored by focusing on a single candidate neural system---the DMN (\cref{ch:study1}) at the whole-brain level (\cref{ch:study2}).

It is a widely held view that the DMN is important in certain types of ongoing thought, such as mind-wandering \cite<see a review from>{SmallwoodSchooler2015}. The aim of \cref{ch:study1} was to explore whether associations between patterns of DMN connectivity and measures of experience yielded unique patterns with distinctive patterns of functional outcomes. In this study we found two reliable neuroexperiential patterns, each with distinct functional outcomes (see \cref{fig:study1:fig4}). Internal connectivity in the mPFC was related to positive habitual experience and was predicted by poor executive control (see \cref{fig:study1:fig2}), suggesting that this may correspond to patterns of executive failure linked to the mind-wandering state \cite{McVay2009}. The relationship between ongoing thought and deficits in executive control could be a result of failure to allocate cognitive resource to an external task. In addition, patterns of PCC-TPJ-mPFC decoupling (see \cref{fig:study1:fig2}), associated with off-task experience, provided a link to better performance on tasks requiring the generation of information.  The continuous content generation associated with an external task is similar to the association between patterns of off-task thought and creativity \cite{Baird2012}. Together, \cref{ch:study1} provided evidence that ongoing thoughts unfold along a set of heterogeneous dimensions. Critically, \cref{ch:study1} explained how the conflicts between the representational and executive failure accounts could be a consequence of different configurations in the DMN.

Since the brain works as a distributed system when engaging in tasks, a natural question following \cref{ch:study1} is how regions other than DMN contribute to the heterogeneity of ongoing thought. Recent works on the hierarchical functional cortical organisation suggest variability in relationships among large-scale networks \cite<e.g.>{Margulies2016}. In \cref{ch:study2}, the focus shifted from the DMN to the whole brain in order to explore the contribution of other brain networks to the heterogeneity of ongoing experience. The identified neuroexperiential components suggested that different patterns of ongoing thought may be linked to distinctive neural hierarchies. For example, experiences characterised as purposeful monologue (see component 1 in \cref{fig:study2:fig2}) were linked to sensory segregation---dissociation between the DMN and unimodal networks---at the whole-brain level. This pattern of decoupling is a well-documented element of ongoing thought \cite<see>{Smallwood2013,SmallwoodFrontiers2013} and is thought to reduce the interference between patterns of self-generated thought and events in the external environment \cite{Murphy2018}. Consistent with the adaptive view on the sensory segregation process, this pattern of experience was linked to better performance on measures of cognition and intelligence (\cref{fig:study2:fig3}). The study also highlighted that dysfunctions within a second hierarchy---the MDN \cite{Duncan2010}---may also make an important contribution to patterns of ongoing thought (see component 3 in \cref{fig:study2:fig2}). Individuals whose thoughts were directed towards their personal concerns had low levels of connectivity within both the ventral and dorsal attention networks, as well as the FPN. Critically, these individuals tended to show poor performance at measures of intelligence and control. This phenotypical pattern provides evidence for the hallmarks of executive failure \cite<i.e.>{McVay2010}. In contrast to data present in \cref{ch:study1}, where a singular system gives rise to different patterns of thoughts and behaviour, the data presented in \cref{ch:study2} shows that heterogeneous patterns of ongoing thought emerge from functional connectivity that reflects previously documented neurocognitive associations.

Multiple overlapping patterns of ongoing thought can be realised by combining experiential data with objective neurocognitive measures. The demonstrations in this thesis suggest that some of the theoretical controversy surrounding the nature of ongoing thought can be explained as relating to distinct patterns of ongoing thought. For example, in both \cref{ch:study1,ch:study2}, we found certain patterns of ongoing experience that have beneficial features (such as better creativity or intelligence) and others with less beneficial correlates (such as lower intelligence or worse executive control). Based on these findings, some of the controversies regarding whether mind-wandering should be considered a failure of executive control \cite<e.g.>{McVay2010,Smallwood2010} result from prior studies lumping together experiential states with different features into a single category. In other words, one important contribution of this thesis is providing a synthesis to resolve the competing theoretical positions of the same phenomenon. The competing elements highlighted in the theoretical accounts can be seen to reflect independent aspects of ongoing thought. The capacity to identify these overlapping patterns of experience is made possible in part because of the use of a biological marker (i.e. functional connectivity) as well as assessments on multiple aspects of ongoing thought and task performance. Moving forward, studies of ongoing thought would, therefore, benefit from measuring multiple dimensions of experience, as well as measures of covert function such as neuroimaging measures, pupillometry \cite{Konishi2017}, or other biological measures \cite{Engert2014}.


\subsection{Integration and segregation in transmodal networks}

A second theme emerging from this thesis lies in the function of transmodal networks. Patterns of heterogeneity emerge through differential patterns of integration and segregation between and within large-scale neural systems. Integration and segregation have been both assumed to be an important principle in brain organisation.
%DMN
For example, hierarchical integration of sensory information is thought to support more abstract aspects of cognition \cite{Mesulam1998}. In contrast, segregating neural systems are thought to provide flexibility in the operations that can be performed \cite{Buckner2013}. A hierarchical organisation implicating both integration and segregation is captured by the primary gradient which stretches from the unimodal to the transmodal networks \cite{Margulies2016}.
%MDN
Other examples of cognitive hierarchies that depend on integration and segregation include the MDN \cite{Duncan2010}. The integrated activity of large-scale network concerned with integration and segregation is important whenever individuals perform complex goal-directed tasks.  Notably, the principal gradient and the MDN are differentiable in terms of the degree of separation between the DMN and FPN. The differences in functional distance indicate how patterns of integration within transmodal cortex is a defining feature of the neurocognitive hierarchies.
%limbic
The other hierarchy emerges from the limbic system. This limbic system hierarchy is composed of visceromotor regions that connect with DMN, and salience network \cite{Kleckner2017}. These authors suggest that this forms an allostatic-interoceptive system, segregated from the unimodal and attention systems. This hierarchy is assumed to emerge because the limbic system can selectively integrate information from systems involved in attention and cognition, as well as those important for emotion and affect.
%short summary on past study
These past studies illustrate that at the core of different neural hierarchies are patterns of integration and segregation between distributed neural regions.

This thesis underscores the implication of the importance of integration and segregation in the neural hierarchies that support patterns of ongoing thought. \cref{ch:study1} demonstrated that the role of the DMN in distinct patterns of ongoing thought emerges because of differences in the integration and segregation within DMN, with spontaneous off-task thought linked to lower connectivity within this system, while the positive habitual thought was linked to integration within the same system. Importantly, this latter pattern of experience was also linked to stronger coupling with a number of regions outside of the DMN including left temporoparietal cortex, left hippocampus/entorhinal cortex, left lateral middle temporal gyrus, and the left pre-SMA. We also found evidence for integration and segregation in \cref{ch:study2}. The pattern of purposeful future planning thought was linked to segregation between DMN and the primary sensory systems. The second pattern of thoughts reflected ongoing thoughts of personal importance and was linked to reduced connectivity (i.e. lower integration) within many regions of attention and control systems. Within both \cref{ch:study1,ch:study2} patterns of ongoing thought were differentiable based on the patterns of integration and segregation between neural systems.

The most powerful evidence for integration and segregation within this thesis is provided by \cref{ch:study3}. This analysis highlighted the integration and segregation in the limbic system as a core determinant of patterns of ongoing thought. The limbic system has been previously argued to form a hierarchy integrating the transmodal system while simultaneously segregating the unimodal system to facilitate attention inhibition \cite{Kleckner2017}. In \cref{ch:study3} we found a pattern of population variation anchored at one end by a highly inter-connected limbic system, integrating with the other transmodal area and segregated from the sensorimotor system. Such segregation pattern was predictive of behaviour that entailed the flexibility of retaining mental content. At the other extreme, the limbic system was highly coupled with neural system and individuals were unable to inhibit their prior mental set. Importantly, we found that this pattern of population variation was linked to patterns of ongoing thought that varied from task-focused thought at one end to personal, habitual content at the other. This analysis not only suggests that the degree of integration and segregation between the limbic system is important for attentional control \cite<i.e.>{Kleckner2017}. Moreover, it also suggests that the degree to which this system is coupled or decoupled from other aspects of the cortex is a primary determinant of whether patterns of ongoing thought are focused on the task, or are instead focused on personally relevant matters in a detailed manner.

Together the three studies presented in this thesis show that at the core of different patterns of ongoing thought are the integration and segregation between neural systems. Moving forward studies should formally consider how patterns of integration and segregation between neural systems can give rise to the heterogeneity of patterns of ongoing thought that make up our daily lives.


\section{Limitations}

% trait vs state
The primary limitation of this work is how it dealt with the temporal elements of cognition. For example, the studies focused exclusively on individual differences within a population rather than a state level of ongoing thought. Studies exploring the associations between static functional connectivity and psychological traits have brought fruitful results to ongoing thought \cite{Smallwood2016,McVay2009,RubyPlos2013}. However, it is import to bear in mind that these studies confound traits with states since a defining feature of patterns of ongoing thought is their intermittent nature. While individual traits allow a way to understand links between cognition and the brain, it remains to be seen whether the patterns discovered at the population level will be applicable in the momentary state.

There are two ways that future studies could provide a more valid temporal perspective on patterns of ongoing thought. One approach would be to measure experience and neural activity on multiple occasions. One potential strategy is collecting multiple examples of experience using online probes while the neural function is recorded. Recently using the same 0-back/1-back task in conjunction with measures of neural function provided by fMRI we found that different dimensions of experience can have unique neural representations \cite{Sormaz2018}. It would be possible to apply CCA to data collected in this manner which would allow neurocognitive patterns to emerge that describe momentary states rather than population variation. A second approach would be to explore the association between patterns of dynamic neural function and ongoing experience. The recent discovery of temporal dynamic using hidden Markov models \cite<HMM;>{Vidaurre2017} demonstrated that time spent in brain states predicts behavioural traits including measures of inhibition control and attention. HMM allow the identification of temporally re-occurring states that are defined by a similarity between neural data across time. It is possible that the application of HMM to neural data would resolve covert patterns of neural function that reflect momentary changes in patterns of ongoing thought.

Another limitation emerges from the statistical technique that was applied in this thesis in terms of both model selection strategy employed. The three studies in the current thesis explored and improved the model selection strategy of SCCA. The analysis in \cref{ch:study1} did not select the hyper-parameters in a data-driven manner. With formal hyper-parameter selection, \cref{ch:study2} is more transparent and data-driven in the model selection. A nested cross-validation scheme was adopted to simultaneously select the hyper-parameter and the final model. With the motivation to construct a pipeline that can be generalised to the basic version of linear CCA, \cref{ch:study3} separates the hyper-parameter selection step and the mode selection. The final canonical correlations of the principle mode improved from 0.28 in \cref{ch:study2} to 0.70 in \cref{ch:study3} with a simpler pipeline. The scope of the current thesis focuses on the psychological question of ongoing thought, hence the two pipelines presented in \cref{ch:study2,ch:study3} are not formally bench-marked on the same data set. Future works on a structurally simple and well-performed pipeline would be important for the application of CCA and its variation on biomedical data.

The other concern is the choice of optimisation target for model selection. The current thesis uses out-of-sample explained variance as the learning target. The rationale is to maximise the potential of predictability in a wider unknown sample with the limited sample size. The alternative choice would be the out-of-sample prediction error, which minimises the mistake when applied to an unknown sample. This thesis did not explore the second option, hence the performance is unknown. These two optimisation targets are asking two fundamentally different questions---explained variance provides a more optimistic view of the model, while prediction error is more conservative. It is uncertain whether the choice of learning target should be question-driven or performance-driven. Again, a bench-marking study would be helpful to clarify the potential of the options.



\section{Future directions}

Before concluding it is worth considering the implications for two specific areas of the study of ongoing thought. Much debate has been around the intermittent disruption caused by experience sampling methods when intending to measure the train of thought in a concurrent task \cite{SmallwoodSchooler2006}. This problem is especially concerning with MDES, where participants spend around a minute to report the thought rather than one or two questions that can be done in seconds. A covert marker would allow studying the ongoing thought while not interrupting the natural flow of thought.

This thesis shows that the variation in whole-brain functional hierarchy potentially supports different types of ongoing thought. If, as this thesis suggests, patterns of integration and segregation in neural activity are important aspects of different features of ongoing thought, then the covert marker could be based on patterns of functional connectivity. However, the calculation of connectivity depends on the processing of time-series data making the determination of rapid temporal changes problematic. This is compounded by the low temporal resolutions of fMRI. The application of magnetoencephalography (MEG) is possibly helpful for the determination of an online marker, given its ability to reveal neural processes at the level of milliseconds superior to fMRI.

In conclusion, the current thesis provided a proof of principle on the utility of whole-brain functional connectivity in exploring ongoing thought. It has the potential to be the covert online marker for spontaneous thought. However, with the current limitation in fMRI temporal resolution, functional connectivity calculation would be the main challenge of such application. MEG is the possible candidate method for understanding the dynamic of ongoing thought and underlying neural pattern.


% ==========================================================================================================

\section{Concluding remarks}
\label{ch:discussion:summary}

This thesis set out to examine the neurocognitive mechanism of ongoing thought and establish the basic component processes to incorporate the heterogeneity of ongoing thought. Three major questions were posed at the start of the thesis. These will now be revisited in light of the work performed.

\textbf{Why does ongoing thought show both costs and benefits?} Ongoing thought is a collective phenomenon with multiple types of experience each with their own associated functional outcomes at the trait level. This thesis suggests that pattern of costs and benefits related to mind-wandering may be usefully conceptualised as characterising overlapping but distinct aspects of the ongoing experience. Further work will be important to understand the underlying mechanisms that explain why these different states emerge.

\textbf{Can functional neural hierarchy explain the heterogeneity?} This thesis demonstrates that ongoing thoughts with different experimental profiles are associated with different neural hierarchy. Further work is suggested to incorporate the neural basis with the ongoing thought profiles at the state level to understand the dynamic of ongoing thought.

\textbf{Is the family resemblance view viable for ongoing thought?} Overall this thesis supports the contention that ongoing thought can be conceived of as a family of experiences with similar and overlapping features. The current thesis finds common component processes that determine population variation. Further work is necessary on the application of these findings at a state level to improve the understanding of the architecture of the component processes.
