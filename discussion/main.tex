\chapter{General discussions}
\chaptermark{Discussion}
\label{ch:discussion}
%\setcounter{equation}{0}
% ==========================================================================================================
Contemporary research on patterns of self-generated thoughts, such as those occurring during states of mind wandering, are riddled with contradictions. For example, the content of on-going thought varies from future-orientated planning thoughts that may help refine personal goals \cite{Medea2016} from negative past concerns that can maintain unpleasant affective states \cite{Killingsworth2010}. Likewise, research has highlighted in tasks that demand continuous external attention \cite{McVayJOEP2009,McVay2012} whereas evidence from studies of creativity and problem solving suggest evidence of a benefit \cite{Smeekens2016,Baird2012}. 

The conflicting evidence above is thought to emerge because of the variability in the nature of on-going thought. The heterogeneous on-going thought may be composed of a set of experiences with overlapping features---the so called family resemblance account of mind-wandering \cite{Smallwood2013, Seli2018}. Building on this assumption, the focus of this thesis was to develop an empirical approach to  sensitive to the possibly similar notions among the heterogeneous thought patters. In other words, there could be multiple patterns of on-going experience that share common and distinct features. 
A second assumption of this thesis is that understanding the categories of experience can be informed by examining the intersection between subjective reports objective measures---in this case patterns in resting state functional connectivity and performance on cognitive tests. One method that is sensitive to both the assumed heterogeneity and the need to constrain self-reported data by objective metrics is SCCA \cite{WittenSCCA2009}---a multivariate approach that measures similarities of linear patterns in two domains of data. The overarching aim of the thesis is to provide evidence in support of the family resemblance view of on-going thought and facilitate future research on the granularity of the interaction of the biological and behavioural domains.

% Old version
% The formalization of on-going thought \cite{SmallwoodSchooler2006} has enriched the scientific understanding of such phenomena. On-going thought is heterogeneous in its definition, functional outcomes, and experiential profile. Some researches focused on on-going thought during a task-free state, such as mind wandering and task-independent thoughts. On the other hand, spontaneous on-going thought also occurs intentionally \cite{SeliTiCS2016}. Researches in mind wandering related poor task performance to on-going thought \cite{McVayJOEP2009, MrazekJoEP2012}, whereas evidences in creativity literature has found the benefit of engaging the internal train of thought \cite{Smeekens2016, Baird2012}. The content of on-going thought varies from future-orientated planning thoughts \cite{Medea2016} to negative past concerns \cite{Killingsworth2010}. The literature has presented the heterogeneous in on-going thought as a conflict. The reason behind the heterogeneity of on-going thought is not well understood. This thesis sought to resolve the conflict literature through understanding the fundamental mechanism of on-going thought. Furthermore, I would like to present the evidence to encompass the variety of on-going thought in a family-resemblances view \cite{Seli2018}. 

% Two theoretical accounts have been proposed to formalize the empirical observations about the on-going thought. The representational account \cite{Smallwood2016} presented on-going thought as a process of generating internal representation that decouples from the external environment. A shift away from the external world helps the mental resource allocate to internal content generation. The representational account focuses to understand the mechanism of on-going thought generation, including both the impairment and benefit that derive from the process. In the executive failure view \cite{McVayJOEP2009}, on-going thought results from the inability of sustaining attention to the external environment. The generation of internal thought reduce the ability to perform attention-demanding tasks. The executive failure view emphasis the on-going thought generated dung demanding external tasks. The context of thought occurrence plays a central role.

% %neural
% Recent studies revealed intrinsic hierarchical organization \cite{Margulies2016} that potentially corresponds to the theoretical accounts of on-going thought. The DMN is optimal for the abstraction of conceptual integration from different experience and modality as well as the segregation of internal and external experience representation in both task and rest \cite{Murphy2018}. The MD network govern the abstract rule reasoning in complex tasks \cite{Duncan2010}. The two networks have different relationship with each other in rest and tasks. The embedded gradient of cortical organization reveals that DMN and MD network are adjacent in the principle gradient while in the third gradient they are opposites \cite{Margulies2016}. The sensory segregation configuration of DMN seems to facilitate the representational account of on-going thought. When DMN and MD network are opposites, it seems to follow the behavior described in executive failure view. A whole brain level study would be the best to study the dynamics among large scale networks. 

% %method and wrap up
% The contradiction in the on-going thought literature is likely to be the result of a one-to-one mapping view on behavior and brain function. Multivariate analysis provides a more naturalistic view when multiple systems are potentially involved. The current thesis employed SCCA \cite{WittenSCCA2009} to find conjoined patterns in resting state functional connectivity and behavioral profiles. The strength of multivariate view captures states that can have opposing features but emerge from a single architecture. I hope to provide evidence to support the family resemblance view of on-going thought and facilitate future research on the granularity of the interaction of the biological and behavioral domains. By laying the foundation of the trait-level finding of on-going thought, the researches will then advance to explore the dynamic, state level details of the generation of on-going thought.

% ====================================================================================================================
\section{Empirical findings}
\label{ch:discussion:results}
% The current thesis provide evidence to support a family resemblances view to integrate the heterogeneous on-going thought with whole brain neural pattern as the basis.  
% what is the take home message of this section? 
% how do each section relate to the goal of the thesis
The current thesis focuses on resolving the heterogeneous features of on-going thought by considering it’s intersection with measures of neural function---in this case the intrinsic functional connectivity at rest. The conjoined decomposition revealed using SCCA various types of on-going thought can be identified at individual level. The results provided evidences to support the family resemblance view of on-going thought through examining the neural mechanism. The findings suggested that on-going thought is a collective term of various types of spontaneous thoughts supported by different neural hierarchy emerge within a population. The following section of this thesis moves on to describe in greater detail the converging evidence of heterogeneity and integration an segregation of transmodal networks. 

\subsection{Heterogeneity}

In the mind-wandering literature, the heterogeneity of related functional outcome  lead to researchers studying different type of on-going thought as independent constructs \cite{SeliTiCS2016} and divergence of theoretical accounts on mind-wandering \cref{ch:intro:accounts}. However, the DMN has been found in preliminary research on unconstrained processes \cite<see>{SmallwoodSchooler2015}. The emergence of a common network related various types of on-going thought is unclear. To explore the mechanism of on-going thought, the ground of the heterogeneity needs to be understood. The current thesis explores the heterogeneous features of on-going thought while considering the intrinsic functional connectivity at rest. \cref{ch:study1} explores the heterogeneity of a single candidate system---the DMN. In \cref{ch:study2}, the heterogeneity is explored in the whole-brain. 

%The conjoined decomposition revealed various types of on-going thought can be identified at individual level and that on-going thought is heterogeneous by its nature. %The competing views are context-dependent cases that emerges from the same neural architecture. In the theoretical aspect, \citeA{Seli2018,SeliTiCS2016} has suggested that external environment unrelated on-going thought (i.e. mind-wandering) can occur with or without intentions. The occurrence under the opposite context leads to a family resemblance view of on-going thought. On-going thought is a heterogeneous constructs that involves a collection of features but not all features are present depending on the context of occurrence. 

The general activity in DMN implied that there is a common system for on-going thought. The variety of functional outcomes related to on-going thoughts could stem from the intra-network connectivity that was not captured through traditional one-to-one mapping view in brain and cognition. The aim of \cref{ch:study1} was to explore how the conjoin patterns of within DMN connectivity and on-going thought relates to the functional outcomes. Two neuro-experiential patterns were relate to distinct functional outcomes. Internal connectivity in mPFC related to positive habitual experience was predicted by poor executive control \cite{McVay2009}. The on-going thought related to deficit in executive control could be a result of failure to allocate resource to the external task. The PCC-TPJ-mPFC negative connection chain conjoining with off-task experience revealed internal generation of information. The continuous content generation associated with external task is similar to the association of mind wandering and creativity \cite{Baird2012}. The data provided evidence that on-going thoughts unfold along a set of heterogeneous dimensions. The conflicts between the representational and executive failure accounts could be a consequences of different configurations in the DMN. 

Since the brain works as a system when engaging in tasks, a natural question following the first study is how regions other than DMN react during on-going thought. Recent works on hierarchical functional cortical organization at rest suggested variability in relationships among large-scale networks \cite{Margulies2016}. In \cref{ch:study2}, the focus shifted from the DMN to the whole brain in order to explore the contribution of other brain networks. The data suggested that among the discovered neuro-experiential components, different neural hierarchies were involved during on-going thoughts. Purposeful monologue associated with sensory segregation---dissociation between the DMN and unimodal networks--- at whole brain level and was predicted by high intelligence. This pattern is well documented in the semantic memory literature, showing that the ability to generate abstract representations relies on the segregation of internal and external representation of the memory\cite{Murphy2018}. The dysfunction of the MD network \cite{Duncan2010} governs the complex reasoning induced by personal importance. This neuro-experimental pattern is predicted by poor performance on intelligent tests. The data suggested the deficit in MD network is the reason of executive failure, rather than DMN. The two neuro-experiential components were both predicted by the same set of abstract reasoning and intelligence test but with distinct underlying neural systems. In contrast to data present in \cref{ch:study1}, where a singular system gives rise to different patterns of thoughts and behavioural, the data presented a similar pattern of behaviour emerged from different functional network hierarchies with their distinct cognitive implication. Again a family resemblance feature was found related to on-going thought. Abstract reasoning is an important cognitive feature that is shared among various type of on-going thoughts. 

%\subsection{Hierarchies in the brain}

%The current thesis has identified different hierarchical systems among large scale networks that facilitates on-going thoughts. Two studies provided empirical data that supports the principle gradient found by \citeA{Margulies2016}. The data in \cref{ch:study1} revealed several satellite regions of DMN including [the name of the regions here], supporting the information integration view of the primary gradient. These supplemental regions are documented in the literature related to integration of in formations in on-going thought\cite{Ellamil2016,Karapanagiotidis2017,Smallwood2016}, and distinguished the content rich experience from mindless thoughts in the current data. The sensory segregation feature in the principle gradient was observed in \cref{ch:study2} through the negtive correlation between DMN and the primary sensory systems. Recent studies has presented empirical evidences supporting the importance of sensory segregation in internal concept representation \cite{Murphy2018,Villena-Gonzalez2018}. The hierarchy of MD network-DMN hierarchy supports abstract rule governing in cognitive demanding context. This pattern is first discovered in task fMRI research \cite{Duncan2010} and been confirmed with new empirical data \cite{Crittenden2016}. The data in \cref{ch:study2} identify the same pattern at rest though the relationship between the dysfunction of MD network and poor performance on intelligence tests. The limbic system forms another hierarchy. The result from \cref{ch:study3} showed the limbic system as highly connected laterally but anti-correlated to the rest of the cortical systems to facilitate flexible inhibition temporally in on-task on-going thought. This pattern resembles the documented function of the limbic system in attention driven, flexible cognition \cite{Kleckner2017}.

\subsection{Integration and segregation in transmodal networks}

Investigations in the common elements in the heterogeneous on-going thought has revealed different component processes. Heterogeneity of ongoing thought is assumed to emerge through difference in the balance of underlying components. The different states of component configurations are assumed to play-out as differences in integration and segregation within architecture. Researchers formed speculations on hierarchical sensory segregation an information integration \cite{Mesulam1998} and the tethering hypothesis on cortical expansion through the macroscopic connectivity \cite{Buckner2013}. This hierarchical integration and segregation view is supported by recent finding on the primary intrinsic functional organization at rest stretches from the unimodal to the transmodal networks \cite{Margulies2016}. Other cognitive hierarchies has been found in task states. Studies on abstract reasoning tasks found a different hierarchy with MD network opposing DMN \cite{Duncan2010}. The limbic system has been found to form a hierarchy facilitating attention inhibition behaviours \cite{Kleckner2017}. The current thesis employed functional connectivity at rest to represent the intrinsic configuration of each individual. The integration and segregation of functional organisation were observed at whole-brain. Population variation of such organisation were associated with different types of on-going thought. 

% study 1
The analysis in \cref{ch:study1} concerns DMN and its role related to on-going thought. The results suggested that within network integration can produce different patterns of experience. While the DMN is internally connected, satellite regions including left temporoparietal cortex, left hippocampus/entorhinal cortex, left lateral middle temporal gyrus, and the left pre-SMA, also coupled with DMN to support memory representation. These supplemental regions distinguished the content rich experience from mindless thoughts in the current data. This observation is cohesive with sensory integration view of the DMN \cite{Margulies2016}, and the different roles DMN integration and segregation \cite<i.e.perception-decoupling and conceptually-guided cognition; >{Murphy2018}. \cref{ch:study1} demonstrated that the local connectivity patterns of the DMN is related to the whole brain. The result implies the importance of understanding the local functional architecture \cite{Braga2015} of transmodal networks in relation to other brain regions. 

% study 2
\cref{ch:study2} also demonstrated he sensory segregation feature in the principle gradient at a whole brain level. The segregation between DMN and the primary sensory systems is associated with good performance on conceptual representation-driven tasks. Recent studies has presented empirical evidences supporting the importance of sensory segregation in internal concept representation \cite{Murphy2018,Villena-Gonzalez2018}. The other segregation pattern was found in the hierarchy of MD network-DMN. The MD hierarchy supports abstract rule governing in cognitive demanding context in task fMRI research \cite{Duncan2010} and been confirmed with new empirical data \cite{Crittenden2016}. The segregation between DMN and MD network worsen the performance on conceptual representation-driven tasks.  

% study 3
\cref{ch:study3} highlighted the integration and segregation in the limbic system. The limbic system has been found to form a hierarchy facilitating attention inhibition behaviours \cite{Kleckner2017}.  Flexibility in cognitive task is associated with the state where the limbic system is internally integrative but segregated from the rest of the brain. The limbic system centric hierarchy is particularly important for determining patterns of off-task on-going thought in the study. 


% old ver:
%The relationship between brain and cognition can be described in very different world view depending on a localist or more whole brain approaches. In the localist view, highly specialized brain regions are mapped to very specific perceptual or cognitive functions, such as FFA and the visual cortex. On the other hand, intra-network relationships facilitate basic level of cognition. For example, researchers formed speculations on hierarchical sensory segregation an information integration \cite{Mesulam1998} and the tethering hypothesis on cortical expansion through the macroscopic connectivity \cite{Buckner2013}. This view is supported by recent finding on the primary intrinsic functional organization at rest stretches from the unimodal to the transmodal networks \cite{Margulies2016}. Other cognitive hierarchies has been found in task states. Studies on abstract reasoning tasks found a different hierarchy with MD network opposing DMN \cite{Duncan2010}. The hierarchical view of functional brain fits the family resemblance view as these hierarchies support very basic and abstract cognitive functions that can be involved in multiple tasks.


%The transmodal networks support a wide variety of cognitive functions at a more abstract level than the unimodal network. The functional organization of the transmodal regions has their specific properties. The current thesis showed that the functional organizations of domain general networks (i.e. limbic network and DMN) are very different when related to the same kind of task. The positive-habitual component found in \cref{ch:study1} showed that when DMN integrated with other regions while internally connected, the configuration is associated with a lack of control, possibly associated with memory representation. The result from \cref{ch:study3} showed that the limbic system is either internally connected or integrating from the cortex when associated with sustained attention. Attention-focused tasks were related to a rich-hub-like integration in DMN and a simultaneous integration and segregation of limbic system. 

%Previous research on affective content of on-going thoughts suggested that emotion-related content is self-generated in response to both external experimental cue and unconstrained thought \cite{Tusche2014}. The study identified common brain region driving the affective content regardless of external cue. The current thesis confirmed emotional thoughts (DMN, \cref{ch:study1}) and stimulus triggered thoughts (Limbic , \cref{ch:study3}) are related to unconstrained whole-brain activity patterns. When investigating unconstrained processes in whole brain, distinct integration/segregation patterns were uncovered. The results implies the importance of understanding the local functional architecture \cite{Braga2015} of transmodal networks in relation to other brain regions. 
 
\subsection{Interim summary and limitations}

The current thesis was a preliminary exploration to the family resemblance processes of on-going thought. Heterogeneity is the centre of debate of past literature on mind wandering. The current thesis expected that understanding the reason of heterogeneity would lead to hints in the underlying mechanism. To advance the current understanding of the heterogeneity in behaviour, the current thesis introduced the intrinsic functional connectivity at rest as the biological basis. SCCA was employed to reveal the conjoined patterns of functional connectivity an behaviour. The evidences converged into two themes that support the goal of the current thesis---heterogeneity an the integration and segregation of transmodal networks. 

The current thesis demonstrated the heterogeneous nature of on-going thought. Multiple types of on-going thoughts were uncovered in \cref{ch:study1,ch:study2} respectively with meaningful associations to cognitive functions such as positive or negative emotion, executive controls and mental representation. This is the first study to uncover different types of spontaneous thoughts in individual data sets. The heterogeneous in self-generated spontaneous thoughts should not be studied as isolated construct but as complimentary aspects of the same phenomena. The theoretical accounts on executive failure and memory representation should be seen as potential component processes rather than conflicts. This conclusion supports the family resemblance view: on-going thought is a set of experiences with overlapping features. 

By adopting a multivariate method and functional connectivity, the current thesis provided a speculation on how the heterogeneous component processes manifest the family resemblance view. The three empirical chapters provide converging evidence on the dynamics of neural hierarchies contributing to different cognitive functions. The function of transmodal regions depended on the whole-brain configurations of integration or segregation. \cref{ch:study1,ch:study2} both demonstrate the sensory integration/segregation of DMN. The segregation of the MD hierarchy was found in \cref{ch:study2} related to demanding executive control function. The limbic system was found segregated from the rest of the brain to sustain attention for changes in the environment. The activity in the neural hierarchy demonstrated the potential common system underlying the on-going thought. The family resemblance lies in the potential ability of transmodal system to both integrate and segregate information from other regions. 

The current thesis focused on individual differences in a population rather than state level of on-going thought. The data sets used in the current analysis focused on resting-state fMRI and individual-level feature on cognitive functions. The benefit of resting state data set is the wide variety of task selections due to the convenience in data collection. The discovery presented in the current thesis is therefore limited to individual level. The two limitations of the thesis are both possibly related to the focus on population variations. Firstly, the current research under-utilise the dynamic nature of resting state by extracting a individual level connectivity profile. Resting state fMRI data has been found to display both trait and state like features \cite{Geerligs2015}. On the other hand, non-biophysical model of the resting state showed changes in the functional connectivity in the resting state \cite{Vidaurre2017}. The second limitation is the unclear structure among the component processes. On-going thought is a dynamic change of states \cite{KucyiNI2017}. Researches on state-related features of on-going thought may provide more insight on the interaction of different component processes.

In conclusion, the current data supports that on-going thought is a family of multiple component processes. Different types of on-going thoughts are relate to DMN because it can be the centre of different integrative/segregative patterns in the whole brain. Similar on-going experience can lead to different functional outcomes depending on the neural hierarchical organisations. The detailed architecture of the component processes would be the potential future research direction. 
 
% \section{Component process account of on-going thought}
% \label{ch:discussion:components}
% %overview here

% The component processing account of on-going thought \cite{SmallwoodSchooler2015} has been effectively examined with a multivariate data decomposition in the current thesis. The most important contribution of the current thesis is a confirmation of the importance of exploring on-going thought through conjoined patterns of two domains of information (e.g. functional connectivity and behavior measures). The component processing account suggested that the combination of components from different domains (e.g. task performance, experience content, and neural functions). Different interactions among a set of components results in the heterogeneity of on-going thought. Recently, the family resemblances view on mind wandering \cite{Seli2018} suggested that different types of on-going thought are held to gather by overlapping subsets of similarities (i.e. family resemblance) rather than a group of all-or-none members. A sub-type of the ongoing thought can be determined by both common and distinct features of other sub-types, thus the members of on-going experiences possess heterogeneous profiles. \citeA{Seli2018} suggest a family-resemblances approach allows the discussions of non-prototypical mind-wandering as neighbouring constructs, rather than isolated or separate subject of study. The current thesis suggested the heterogeneity in on-going thought can integrate the family resemblance framework to extend the discussion from mind-wandering to the general on-going thought regardless of the involvement of external environment. 

% The current thesis utilized SCCA---a multivariate analysis to reveal the conjoined patterns of two domains of the data. With its potential to uncover more than one component, SCCA helped identify component processes of on-going thought and converging evidence to support the component processes account of on-going thought. In \cref{ch:study1}, the DMN gives rise to different types of experience. This suggested  a single systems can have multiple states. In \cref{ch:study3}, individual differences in tasks are predictive of the different components of thought. The experience sampling questions predicting the neuro-cognitive component is a combination of two thought domains, showing that a single dimension can organise multiple components of experience. Most importantly, the data showed the benefit of adopting a multivariate approach to study cognition. Human behaviour is not likely to be driven from one single isolated component, but a complex system. A function localization approach can introduce unexplainable variance due to the exclusion of other on-going cognitive process. 

% Data presented in the current thesis was a preliminary exploration to the component process account of on-going thought. The data sets used in the analysis focused on resting-state fMRI and individual level trait like feature on cognitive functions. The benefit of resting state data set is the wide variety of task selections due to the convenience in data collection. The discovery presented in the current thesis is therefore limited to individual level. Firstly, the current research under-untilize the dynamic nature of resting state by extracting a individual level connectivity profile. This is not suggesting resting state should not be used in such way. Resting state fMRI data has been found to display both trait and state like features \cite{Geerligs2015}. On the other hand, non-biophysical model of the resting state showed changes in the functional connectivity in the resting state \cite{Vidaurre2017}. The second limitation lies in the dynamic nature of on-going thought \cite{KucyiNI2017}. The structure of the components will be clearer with more understanding of the state-related features of on-going thought

% ==========================================================================================================

\section{Future directions}
\label{ch:discussion:future}
The current thesis aims to understand the mechanistic view of the on-going thought. Through examination of the component process account with SCCA, we established that on-going thought is heterogeneous in its nature, and a family resemblance view resolves the conflicts in categorising and defining the topic of interest. I would like to propose suggestions on methodological and theoretical research directions to further explore the established findings of the current thesis.

\subsection{Whole brain functional connectivity as a convert marker}
This thesis provided evidence of the neuro-cognitive component processes of on-going thought derive its heterogeneity at individual trait level. The conjoined decomposition of resting state functional connectivity and behaviour discovered hierarchical structures in large-scale network  that supports different types of on-going thought. The data from \cref{ch:study1} showed that regions out side of the DMN are involved in positive-habitual experience but not off-task thoughts, implicating the importance of whole brain activity for disentangle the underlying mechanism of on-going thoughts. Therefore, a whole brain parcellation was adopted in the subsequent studies. \cref{ch:study2} demonstrated structurally dissimilar neuro-experiential components can be associated with similar cognitive functions. 

The current data supported related research on whole brain functional connectivity as potential covert marker of trait level features. The large-scale networks demonstrated degeneracy when considering whole-brain activity. The result in \cref{ch:study2} suggested functional brain can be degenerate. Degeneracy describes that structurally dissimilar biological modules can perform similar functions under certain conditions, but still demonstrate unique functions in other conditions \cite{Buckner2013}. Large scale networks discovered at group level showed resemblance to network organisation at individual level \cite{Finn2015}. Task potency comparing the difference of functional connectivity between task and rest has been proven to predict age difference in context switching related tasks \cite{Chauvin2018}. Study in the future should adopt whole brain functional connectivity measures, such as ROI-based functional connectivity, or connectivity-driven matrices such as diffusion embedded gradients \cite{Margulies2016,Marquand2017}, for complex cognitive process such as on-going thought.

\subsection{SCCA}

The three studies in the current thesis explored and improved the model selection strategy of SCCA. The analysis in \cref{ch:study1} did not select the hyper-parameters in a data-driven manner. With formal hyper-parameter selection, \cref{ch:study2} is more transparent and data-driven in the model selection. A nested cross-validation scheme was adopted to simultaneously select the hyper-parameter and the final model. With the motivation to construct a pipeline that can be generalised to the basic version of linear CCA, \cref{ch:study3} separate the hyper-parameter selection step and the mode selection. The final canonical correlations of the principle mode improved from 0.28 in \cref{ch:study2} to 0.70 in \cref{ch:study3} with a simpler pipeline. The scope of the current thesis focuses on the psychological question of on-going thought, hence the two pipelines presented in \cref{ch:study2,ch:study3} are not formally bench-marked on the same data set. Future work a structurally simple and well-performed pipeline would be important for the application of CCA and its variation on biomedical data.

The other concern is the choice of optimisation target for model selection. The current thesis uses out-of-sample explained variance as the learning target. The rationale is to maximise the potential of predictability in a wider unknown sample with the limited sample size. The alternative choice would be the out-of-sample prediction error, which minimise the mistake when applied to a unknown sample. This thesis did not explore the second option, hence the performance is unknown. These two optimisation targets are asking two fundamentally different questions---explained variance provides a more optimistic view of the model, while prediction error is more conservative. It is uncertain whether the choice of learning target should be question-driven or performance-driven. Again, a bench-marking study would be helpful to clarify the potential of the options. 

\subsection{Shift to states rather than traits}

The current thesis studies on-going thought at individual level due to the trait-level behavioural measures of the selected data sets. The data cannot separate the influences of states and traits on our observed components. Individual traits has been explore in on-going thought \cite{Smallwood2016,McVay2009,RubyPlos2013} and intrinsic neural organisation \cite{Smith2015}. Nonetheless, whether the observed patterns in the current thesis exist at state domain is unclear. Insight into the change of state in on-going thought could be achieved by a focus on dynamic rather than static connectivity \cite{KucyiNI2017}. Sliding window analysis \cite{Chang2010} or hidden Markov models \cite{Vidaurre2017} on resting state data could provide complementary information to the individual level observations, such as the structure of trait-level component processes or state specific patterns would not be captured in the current data.

The change of state in on-going thought can be explore in task-fMRI experiments. The ideal task design would be collecting rich behavioural profiles yielding multivariate responses. A recent example is the fMRI version of the 0-back/1-back experience sampling task (see \cref{study1:method:d}) used by Sormaz et al. (2018) to explore the activation of DMN in tasks. The design of this fMRI version prolong the post-probe response time in order to prevent the task from under-/over-run in the scanner. This compensation results in less experience probes being collected. To improve the current design while matching the task and scanner run time, the number of presented trials should be constantly re-evaluated and generate suitable number of trials. The improved version will fit the scanner run-time and be more similar to the laboratory version. The current n-back task does not synchronise the presentation on screen and the TR. A improved design would coordinate the presentation on screen with the TR, thus the on-screen information will be useful variable to account for the environment during the occurrence of on-going thought.

Application of CCA on the neural activity and related experiential/cognitive measures could examine the neuro-cognitive pattern at state level. The challenge will not be in the CCA model selection itself but in collecting the suitable feature for CCA to evaluate. The task will need to collect sufficient rich neural and behavioural information at a short time window. If the analysis aims to retrieve neuro-experiential components as the current thesis, finding the balance between the frequency of thought probe presentation and sufficient number of TR for functional connectivity calculation will be the main challenge.

% ==========================================================================================================

\section{Concluding remarks}
\label{ch:discussion:summary}
The functional outcome of different type of on-going thought is well documented in the literature. However, the underlying mechanism of on-going thought and the reason of its heterogeneity remains a mystery. On-going thought during unconstrained state is associated with the emergence of DMN. Recent discovery of neural hierarchy in unconstrained processes reveals a network can be involve in multiple configuration. DMN emerges demanding task and internal content representation each with their unique whole brain configuration. These evidence suggested unconstrained processes such as on-going thought would be heterogeneous with the the wide variation of neural hierarchical organisation. The current thesis aims to uncover the neuro-cognitive mechanism of on-going thought and establish the basic component processes to incorporate the heterogeneity of on-going thought. The evidences from the current thesis suggests on-going thought is a family of heterogeneous member with overlapping similarities. Distinct whole brain neural hierarchies support different functional associations of self-generated on-going thought. DMN and the limbic system were found to interact with the whole brain in different manner while both derived from internally generated thought. The current thesis provided individual differences evidence suggesting that the on-going thought is a family of component processes that generates heterogeneous profiles. However, the structure of the component processes remains unknown. Future studies on on-going thought or unconstrained process should consider a multivariate whole-brain approach to consider the hierarchy in cortical organisation rather than individual regions. Investigation of transient on-going thought will be the future direction of research for a deeper understanding of the structure behind the component processes. 