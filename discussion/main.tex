\chapter{General Discussion}
\chaptermark{Discussion}
\label{ch:discussion}
%\setcounter{equation}{0}
% ==========================================================================================================
\newpage
% ==========================================================================================================

%Recap the intro
The formalization of on-going thought \cite{SmallwoodSchooler2006} has enriched the scientific understanding of such phenomena. On-going thought is heterogeneous in its definition, functional outcomes, and experiential profile. Some researches focused on on-going thought during a task-free state, such as mind wandering and task-independent thoughts. On the other hand, spontaneous on-going thought also occurs intentionally. Researches in mind wandering related poor task performance to on-going thought, whereas evidences in creativity literature has found the benefit of engaging the internal train of thought. The content of on-going thought varies from future-orientated happy thoughts to negative past concerns. The literature has presented the heterogeneous in on-going thought as a conflict. The reason behind the heterogeneity of on-going thought is not well understood. This thesis sought to resolve the conflict literature through understanding the fundamental mechanism of on-going thought. Furthermore, I would like to present the evidence to encompass the variety of on-going thought in a family-resemblances view \cite{Seli2018}. 

Two theoretical accounts have been proposed to formalize the empirical observations about the on-going thought. The representational account \cite{Smallwood2016} presented on-going thought as a process of generating internal representation that decouples from the external environment. A shift away from the external world helps the mental resource allocate to internal content generation. The representational account focuses to understand the mechanism of on-going thought generation, including both the impairment and benefit that derive from the process. In the executive failure view \cite{McVayJOEP2009}, on-going thought results from the inability of sustaining attention to the external environment. The generation of internal thought reduce the ability to perform attention-demanding tasks. The executive failure view emphasis the on-going thought generated during demanding external tasks. The context of thought occurrence plays a central role.

%neural
Recent studies revealed intrinsic hierarchical organization that potentially corresponds to the theoretical accounts of on-going thought. The DMN is optimal for the abstraction of conceptual integration from different experience and modality as well as the segregation of internal and external experience representation in both task and rest \cite{Murphy2018}. The MDN govern the abstract rule reasoning in complex tasks \cite{Duncan2010}. The two networks have different relationship with each other in rest and tasks. The embedded gradient of cortical organization reveals that DMN and MDN are adjacent in the principle gradient while in the third gradient they are opposites \cite{Margulies2016}. The sensory segregation configuration of DMN seems to facilitate the representational account of on-going thought. When DMN and MDN are opposites, it seems to follow the behavior described in executive failure view. A whole brain level study would be the best to study the dynamics among large scale networks. 

%method and wrap up
The contradiction in the on-going thought literature is likely to be the result of a one-to-one mapping view on behavior and brain function. Multivariate analysis provides a more naturalistic view when multiple systems are potentially involved. The current thesis employed SCCA \cite{WittenSCCA2009} to find conjoined patterns in resting state functional connectivity and behavioral profiles. The strength of multivariate view captures states that can have opposing features but emerge from a single architecture. I hope to provide evidence to support the family resemblance view of on-going thought and facilitate future research on the fine-grained details of the interaction of the biological and behavioral domain of the findings. By laying the foundation of the trait-level finding of on-going thought, the researches will then advance to explore the state level details of the generation of on-going thought.

% ====================================================================================================================
\section{Empirical findings}
\label{ch:discussion:results}
\subsection{Heterogeneity}
The current thesis focuses on exploring the heterogeneous features of on-going thought while considering the intrinsic functional connectivity at rest. I have come to the conclusion that on-going thought is heterogeneous by its nature. The competing views are context-dependent cases that emerges from the same neural architecture. In the theoretical aspect, Seli and friends \cite{Seli2018,SeliTiCS2016} has suggested that external environment unrelated on-going thought (i.e. mind-wandering) can occur with or without intentions. The occurrence under the opposite context leads to a family resemblance view of on-going thought. On-going thought is a heterogeneous constructs that involves a collection of features but not all features are present depending on the context of occurrence. In the current thesis, I present two piece of converging evidences to show case the family resemblance views.
   
Firstly I would like to discuss the multiple types of on-going thought. In study 1, different configurations of the DMN have been found to relate to different type of behaviors. 
\subsection{Hierarchies in the brain}

\subsection{Integration/segregation in transmodal networks }
% ==========================================================================================================

\section{Component process account}
\label{ch:discussion:components}
%over view here

\subsection{Evidence in the current thesis}


\subsection{Limitations}

% ==========================================================================================================

\section{Future directions}
\label{ch:discussion:future}


\subsection{Utility of functional connectivity as covert marker}

\subsection{SCCA}

\subsection{Shift to states rather than traits}


% ==========================================================================================================

\section{Summary}
\label{ch:discussion:summary}
