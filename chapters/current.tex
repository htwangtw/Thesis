\chapter{The Current Thesis}
\label{ch:cur}
\chaptermark{Current Thesis}
% ==========================================================================================================
In the current PhD, we would like to explore the numerous thoughts to describe human consciousness experience and establish a framework that incorporate the heterogeneity of the mind-wandering experience. The wide range of functional outcomes and contents of the mind-wandering state is perplexing as often time, those features are contradictory against each other. These studies indicated that the mind-wandering state is a heterogeneous collection of experiences that each type of the mind-wandering experiences has its own underlying cognitive mechanism and contextual representation (Smallwood \& Andrews-Hanna, 2013). The current difficulty of advancing the mind-wandering research is that, although the related cognitive and experience phenotypes has been found, we are unsure about the mechanism uniting those features. Moreover, the integrative nature of higher level cognition makes it difficult to identify the contribution of different processes to the mind-wandering state. This PhD aims to advance the understanding of the identified cognitive processes and find the quantifiable features of the cognitive measures that can infer the mind-wandering states. With the combination of MDES that accesses the heterogeneity of the content, cognitive phenotypes and the neuroimaging data, this PhD aims to establish the ontology of spontaneous thoughts and lays a solid foundation for future studies in consciousness. 