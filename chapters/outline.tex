\chapter{Outline}
\label{ch:outline}
\chaptermark{Outline}

% ==========================================================================================================
\newpage
% intro
\lorem

\lorem

% ==========================================================================================================
\section{Outline}
This chapter provides a brief outline of the motivation and objective of each chapter.

% ==========================================================================================================
\section{Literature Review}

Previous research on mind wandering is outlined, starting with the conflicts in the literature related to functional outcomes. Two neural hierarchy accounts were presented to untangle the mechanism of mind-wandering. Finally, the research techniques in mind wandering research are introduced, including behavioural measures and resting state fMRI methods. 

% ==========================================================================================================
\section{Canonical Correlation Analysis}

Canonical correlation analysis is introduced as the main method of the thesis. This review focuses the potential applications in neuroimaging research. The features, applications of this multivariate method are outlined, followed by discussion on the method's the interpretations and limitations. 

% ==========================================================================================================
\section{Current Thesis}

The research question is presented, highlighting the possible research approach provided by a specific variation of CCA – sparse CCA. 

% ==========================================================================================================
\section{Dimensions of Experience}

Heterogeneity of mind wandering leads to conflicts in its functional outcome in behavioural study. Default mode network is commonly associated with the emergence of mind wandering. Sparse CCA conjointly decomposed functional connectivity patterns of DMN and thought reports, revealing unique neuro-experiential components. The study then revealed that the neuro-experiential components each associate with unique cognitive task measures. The various connectivity configuration within DMN is associated with different types of mind wandering and their specific functional outcomes.

% ==========================================================================================================
\section{Patterns of Thought}

Unconstrained cognitive processes have two faces. The representational account argues that the primary sensory brain region decoupling from the DMN to facilitate memory representation; whereas the executive failure account shows lapses in attention is related to the demand to attention system and activation of DMN. Mind wandering is an unconstrained cognitive state, therefore we used sparse CCA to extract related whole brain functional connectivity patterns, profiling the neuro-experiential components of unconstrained cognitive processes. Examining the association between demanding cognitive tasks and neuro-experiential components, the study revealed evidence supporting both the representational and executive failure accounts.

% ==========================================================================================================
\section{Study 3}

\lorem

% ==========================================================================================================
\section{General Discussion}

The overarching themes of the thesis are discussed and linked to specific results throughout the thesis. Future research directions are inspired by the findings and limitations of the current thesis.

% ==========================================================================================================
