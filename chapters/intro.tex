\chapter{Introduction}
\label{ch:intro}
\chaptermark{Introduction}

%\setcounter{equation}{0}
% ==========================================================================================================
\newpage

% ==========================================================================================================
\section{Mind wandering as a Heterogeneous phenomenon}

\subsection{Heterogeneity in definitions}
In past years, mind wandering has been studied in a variety of related psychological domains, such as cognition, emotion, and neuroscience. Various lines of research have addressed the basic phenomenal characteristics of mind wandering---
\begin{quote}
    a shift in the contents of thought away from an ongoing task and/or from events in the external environment to self-generated thoughts and feelings \cite{SmallwoodSchooler2006,SmallwoodSchooler2015}.
\end{quote}
Browadly speaking, researchers approach the study of mind wandering in two different ways: why and how human mind wanders. The two approaches can lead to different foucses. For instance, when studying mind wandering while performing a task, the relationship between the task performance and mind wandering would be the focus of researchers interesed in the `why'; When the research interest is in the `how', the mechnism behind mind wandering's occurence will be the research topic. The two apporaches lead to a conflict of definition in mind wandering literature.

Mind wandering can occur with or without intention \cite{SeliTiCS2016}. Some researches treated mind wandering as an unintentional behaviour. This assumption is most commonly found when the study explicitly instruct the participant to perform a task. As a natural course of consequences, the participant would report the time that they are not focusing on the task as `mind wandering'. The mind wandering period is not intended in the design, hence this leads to the assumption of all mind wandering are unintended. The assumption of all mind wandering states are unintended dismissed the possibility of intentional mind wandering. The participant can, however, intentionally mind wander if they lack a motivation to engage in the experiment. When a simple Yes/No questions is asked, the response does not allow to infer the reason behind the occurrence of mind wandering. In day-to-day life, we are likely to be instructed to perform a certain task such as working, studying, or anything that has a external indicator regarding the person's performance. The lapse of attention is likely to be associated with the poor performance caught on site.

Mind wandering with intention can occur in various scenarios. When participants are asked about the nature of mind wandering period, a proportion of the period reports to be intentional. The reason behind intentional mind wandering in laboratory scenario could be the lack of motivation to complete the task, or the task is not mentally demanding to have all the attention resource allocated to the task. The occurrence of intended and unintended mind wandering can also be down to individual differences.
The intended mind wandering can be seen as a flexible, hyper active state.

\subsection{Heterogeneity in functional outcomes}

%% cost
%executive
\cite{McVayJOEP2009}
\cite{MrazekJoEP2012}

%Reading
\cite{Smallwood2008}
\cite{McVay2012}
\cite{Unsworth2013}

%mood
\cite{Killingsworth2010}
\cite{Smallwood2007}

%%Benefit
%creativity
\cite{Baird2012}
\cite{Smeekens2016}

%mood
\cite{RubyPlos2013}
\cite{Poerio2016}

%future planning
\cite{Medea2016}
\cite{DArgembeau2006}


\subsection{Heterogeneity in experiential profiles }
Investigations of mind wandering use self-report to understand the content of thoughts. The content of mind wandering has a wide variety of topics and modality.
% Temporal focus
\cite{Baird2011}
\cite{Smallwood2011}

% PCA studies from the past
\cite{RubyFP2013}
\cite{RubyPlos2013}
\cite{Gorgolewski2014}
\cite{Smallwood2016}


% ==========================================================================================================

\section{Theoretical accounts of Mind Wandering}
\subsection{Executive failure account}
% (Why mind wandering happens)
\cite{Kane2012}
\cite{McVayJOEP2009}

% Attention
\cite{McVayJOEP2009}
\cite{MrazekJoEP2012}
\cite{Weissman2006}

% Mood
\cite{Killingsworth2010}
\cite{Smallwood2007}

\subsection{Representational account}
% (How mind wandering happens)
% Content of thoughts relies on episodic/semantic memory and integrations of the sensory system.

\cite{Binder2009}
\cite{Gusnard2001}

% Memory
\cite{Smallwood2016}
\cite{Karapanagiotidis2017}

% Modality
\cite{Schooler2011}
\cite{Buckner2013}


% ==========================================================================================================
\section{Neural hierarchies}

\subsection{Historical perspective}
%function specialisation
\cite{Kanwisher2010}
% network
\cite{Sporns2014}
\cite{Mittner2016}
\cite{SmallwoodFrontiers2013}
% gradient
\cite{Margulies2016}

\subsection{Abstract rules governing}
\cite{Duncan2010}
\cite{Fox2005}
\cite{Weissman2006}

\subsection{Sensory integration}
\cite{Mesulam1998}
\cite{Villena-Gonzalez2018}
\cite{Murphy2018}
\cite{VatanseverPNAS2017}

\subsection{Compensation}
\cite{VatanseverPNAS2017}
\cite{Crittenden2015}
\cite{Crittenden2016}



% ==========================================================================================================
\section{Set of considerations for better accounts of cognition}


\begin{figure}[H]
	\centering
	\includegraphics[width=0.8\textwidth]{chapters/img/thesisfig1.png}
	\caption{Schematic of the thesis}
	\label{fig:thesis:fig1}
\end{figure}

% ==========================================================================================================
\section{Mind wandering measuring methods and current progress}

\subsection{Content of experience}

\cite{Medea2016}
\cite{RubyPlos2013}
\cite{Smallwood2016}

\subsection{Method of acquiring experience}
\cite{Christoff2009}
\cite{Konishi2015}

\cite{Smallwood2011}
\cite{Gorgolewski2014}

\subsection{Method of describing the neural organisation using CCA}
% ==========================================================================================================
\section{Summary}

% ==========================================================================================================
