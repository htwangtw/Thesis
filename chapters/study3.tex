\chapter{Study 3}
\chaptermark{Study 3}
\label{ch:study3}
%\setcounter{equation}{0}

\newpage
%Abstract


% ====================================================================================================================
\section{Introduction}
\label{study3:intro}

% ==========================================================================================================

\section{Method}
\label{study3:method}
\subsection{Participants}
\label{study3:method:a}
Two hundred and seven healthy participants were recruited from the University of York (132 females, 65 males; age range = 18–-31 years, \textit{M} = 20.21, \textit{SD} = 2.36). 
This analysis included two data sets with some shared measurements and same MRI protocol as \ref{ch:study1}. 
Participants were right-handed native English speakers with normal or corrected-to-normal vision and no history of psychiatric or neurological illness. Participants underwent MRI scanning, completed an 1-hr online questionnaire. The first cohort is identical to the sample in \ref{ch:study1}. Participant attended three 
(165 participants; 99 females, 66 males; age range = 18–-31 years, \textit{M} = 20.43, \textit{SD} = 2.63) 2-hr behavioral testing sessions to complete a battery of cognitive tasks. 
The second cohort (42 participants; 33 females, 9 males; age range = 18–-23 years, \textit{M} = 19.79, \textit{SD} = 1.37) underwent two 2-hr behavioural testing sessions to complete a battery of cognitive tasks. The behavioural sessions took place within a week of the scan. Ten participants were excluded from the multivariate pattern analysis because they failed to complete all of the behavioural testing sessions. In total, 197 participants (126 females, 71 males; age range = 18–-31 years, \textit{M} = 20.11, \textit{SD} = 2.24) were included in the multivariate pattern analysis and the comparison with cognitive performance. Participants were rewarded with either a payment of \pounds 10 per hour or a commensurate amount of course credit. All participants provided written consent prior to the fMRI session and the first behavioural testing session. Approval for the study was obtained from the ethics committee of the University of York Department of Psychology and the University of York Neuroimaging Centre.

\subsection{MRI acquisition}
\label{study3:method:b}
The MRI acquisition protocol was identical to Chapter \ref{ch:study1}. Please refer to section \ref{study1:method:b} for details.

\subsection{Resting state data preprocessing}
\label{study3:method:c}

All preprocessing and denoising steps for the MRI data were carried out using the SPM software package (Version 12.0) and Conn functional connectivity toolbox (Version 17.f), based on the MATLAB platform (Version 17.a). The first three functional volumes were removed in order to achieve steady state magnetisation. The remaining data was first corrected for motion using six degrees of freedom (x, y, z translations and rotations), and adjusted for differences in slice-time. Subsequently, the high-resolution structural images were co-registered to the mean functional image via rigid-body transformation, segmented into grey/white matter and cerebrospinal fluid probability maps, and all functional volumes were spatially normalized to Montreal Neurological Institute (MNI) space using the segmented images and a priori templates. This indirect procedure utilizes the unified segmentation–normalization framework, which combines tissue segmentation, bias correction, and spatial normalization in a single unified model. No smoothing was employed, complying with recent studies that report the negative influence of this procedure on the construction of connectivity matrices analysis. 

Moreover, a growing body of literature indicates the potential influence of participant motion inside the scanner on the subsequent estimates of functional connectivity. In order to ensure that motion and other artefacts did not confound our data, we have employed an extensive motion-correction procedure and denoising steps, comparable to those reported in the literature. In addition to the removal of six realignment parameters and their second-order derivatives using the general linear model (GLM), a linear detrending term was applied as well as the CompCor method that removed five principle components of the signal from white matter and cerebrospinal fluid. Moreover, the volumes affected by motion were identified and scrubbed based on the conservative settings of motion greater than 0.5 mm and global signal changes larger than z = 3. Though recent reports suggest the ability of global signal regression to account for head motion, it is also known to introduce spurious anti-correlations, and was thus not utilised in our analysis. Finally, a band-pass filter between 0.009 Hz and 0.08 Hz was employed in order to focus on low frequency fluctuations.

\subsection{ROI-ROI functional connectivity.}
\label{study3:method:d}
We adopted a set of 57 regions based on the Yeo 7 networks. We split the networks into two hemispheres and extracted clusters. Two voxels are considered connected only if they are adjacent within the same x, y, or z direction. This yielded 57 clusters from the Yeo 7 networks parcellation. The implementation of spatial clusters extraction was retrieved from python library Nilearn \cite[ \url{http://nilearn.github.io/}, version 0.3.1]{Abraham2014}. Fully-connected, undirected and weighted matrices of bivariate correlation coefficients (Pearson's r) were constructed for each participant using the average BOLD signal time series obtained from all the 57 ROIs described above. The off-diagonal of each correlation matrix contained 1596 unique region-region connection strengths (i.e., the upper or lower triangle of the network covariance matrix). This approach provided a measure of connection strength of the whole brain for each participant. Finally, Fisher’s r-to-z transformation was applied to each network covariance matrix. 

\subsection{Behavioural data}
\label{study3:method:e}

\subsubsection{Cognitive tasks}
We selected X cognitive tasks that are common across the two cohorts and related to the previous literature on mind wandering. The selected tasks includes blah blah. Please refer to Appendix \ref{appendix:study1:subsection2} for the detailed description of the tasks.

\subsubsection{Experience sampling}
Please refer to Chapter \ref{ch:study1} section \ref{study1:method:d} for the experience sampling data collection and Table \ref{tab:study1:1} for the detailed questions. 

We did a PCA on the same thought probes to understand the relation within the reports.


\subsection{Multivariate pattern analysis}
\label{study3:method:f}
\subsubsection{SCCA}
We performed a sparse canonical correlation analysis \cite<SCCA; see>{Hastie2015}%{ Hastie, Tibshirani, \& Wainwright, 2015} 
on the functional connectomes and the cognitive tasks, to yield latent components that reflect multivariate patterns across neural organisation and cognition \cite<For similar application, see>{Wang2018}%Wang et al., 2017). 
SCCA maximised the linear correlation between the low-rank projections of two sets of multivatiate data sets with sparse model to regularise the decomposition solutions a process that helps maximise the interpretability of the results. The regularisation function of choice is L1 penalty, which produces ‘sparse’ coefficients, meaning that the canonical vectors (i.e., translating from full variables to a data matrix’s low-rank components of variation) will contain a number of exactly zero elements. L1 regularisation conducted (i) feature selection (i.e., select only relevant components) and (ii) model estimation (i.e., determine what combination of components best disentangles the neuro-cognitive relationship) in an identical process. This way we handle adverse behaviours of classical linear models in high-dimensional data. A reliable and robust open-source implementation of the SCCA method was retrieved as R package from CRAN 
\cite<PMA, penalized multivariate analysis, version 1.0.9>{Witten2009a}%(PMA, penalized multivariate analysis, version 1.0.9, Witten, Tibshirani, \& Hastie, 2009).
The amount of L1 penalty for the functional connectomes and cognitive task performance were chosen by cross-validation. The procedure is described below. 

\subsubsection{Model Selection}
The model selection process was conducted with two parts: 1) hyperparameter selection 2) mode selection. For the hyperparameter, L1 penalties for the left and right hand side, we performed a grid search on all possible penalty coefficients combination with cross validation on the explained variance of the first mode. The objective is determined by the best out-of-sample explained variance of the first mode. We decompose the full dataset with the selected hyper-parameters. Permutation tests with family-wise error correction \cite{Smith2015} was applied to access the mode(s) that occure above chance. 

\subsection{Group analysis}
\label{study3:method:g}

To determine how patterns of unconstrained neuro-cognitive activity related to performance on the self-report experience, we conducted an independent statistical analysis on the identical subjects. A Type III multivariate multiple regression with Pillai's trace test was applied to 5 individual scores for each of the latent components describing the neuro-cognitive mechanism from the SCCA were the independent variables, and the 13 measures from MDES were the dependent variables that we hoped to described by the linear combination of the latent components. Pillai's trace test is considered to be the most powerful and robust statistic for general use \cite{Huberty2006}.
The p-values reported were based on Bonferroni correction. The analysis was conducted in R (version 3.3.1). The multivariate multiple regression was conducted in R (version 3.3.1) using function ‘Manova’ in R package ‘car’ (companion to applied regression, version 2.1–5).


% ==========================================================================================================

\section{Results}
\label{study3:results}

\subsection{Determining constituent category of cognitive functions}
% grid search results
% permutation test results
% explain the significant components

\subsection{The relationship between neuro-cognitive components and self-reports on thoughts}
% multivariate main effect
% univariate main effects shows resemblance to the PCA

% ==========================================================================================================

\section{Discussion}
\label{study3:discussion}
