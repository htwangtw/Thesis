% Line spacing tweaks
% UoY thesis demands 1.5x but sometimes we need to drop to 1.0x in tables, algorithms, and other awkward elements.
% Call \linespacesmall before the table, then \linespacenormal after the table.
\def\linespacesmall{
\renewcommand{\baselinestretch}{1.0}
}
\def\linespacenormal{
\renewcommand{\baselinestretch}{1.5}
}

% declaration text
% full references in text are all hand-edited. Package bibentry and hyperref have an unsolved conflict. To enable hyperref, I chose not to use bibentry.
\def\contentdeclaration{

I, Hao-Ting Wang, declare that this thesis is a presentation of original work and I am the sole author. I undertook the research at University of York during \yearstartedtext{} -- \yearendedtext{}, under the joint supervision of Professor Jonathan Smallwood and Professor Elizabeth Jefferies. This work has not previously been presented for an award at this, or any other, University. All sources are acknowledged as References.

Some parts of this thesis have been published in peer-reviewed journals or is currently under preparation for publication. Author contributions are noted at the start of each chapter.  

\begin{itemize}
    \item \cref{ch:methods}: Wang, H.-T., Smallwood, J., Satterthwaite, T. D., Bassett, D. S., \& Bzdok, D. (2018). Finding the needle in high dimensions: A tutorial on CCA in biomedicine. Manuscript preparing for publication.	
	\item \cref{ch:study1}: Wang, H.-T., Poerio, G. L., Murphy, C. E., Bzdok, D., Jefferies, E., \& Smallwood, J.(2018). Dimensions of Experience: Exploring the Heterogeneity of the Wandering Mind. \textit{Psychological Science}, \textit{29} (1), 56--71. \url{doi: 10.1177/0956797617728727}
	\item \cref{ch:study2}: Wang, H.-T., Bzdok, D., Margulies, D. S., Craddock, C., Milham, M., Jefferies, E., \& Smallwood, J.(2018). Patterns of thought: Population variation in the associations between large-scale network organisation and self-reported experiences at rest. \textit{NeuroImage}, \textit{176} (1), 518--527. doi: \url{10.1016/j.neuroimage.2018.04.064}
\end{itemize}

}

% acknowledgements text
\def\contentacknowledgement{

The most exciting chapter of my academic journey would not be possible without my supervisors Prof. Jonathan Smallwood and Prof. Elizabeth Jefferies. Jonny and Beth are the kindest human I have ever known. Especially, I cannot be more thankful for Jonny's trust in my ability to do this PhD when I was doubtful of my own ability. That offer really changed my life. Their help and support in both science and life have made this PhD a truly rewarding experience. I learned from them to be a better scientist and also a good person. 

I express my gratitude to Prof. Dr. Dr. Danilo Bzdok for the guidance and discussions on the canonical correlation analysis project. Danilo has influenced my scientific thinking more than I can give him credit for here. 

I thank my thesis  advisory  panel, Dr. Tom Hartley, Dr. Aidan Horner, and Dr. Cade McCall for providing a friendly environment to discuss science and my skill development. The meetings were truly a pleasure! 

All the research projects presented here will not be possible without the current and past members of semantics and thought lab. Their contributions lie not just in the research, and also moral supports on my personal life. Good colleagues like them are difficult to come by. 

Finally, thanks to all my family and friends for all their assistance. Special mentions to Thomas Hardman, whose love, support and bad taste on YouTube videos that has brought me infinite joy. To Rebecca Jones for all her supports during this three years through food and friendship. 

}

% acronyms and abbreviations text
\def\contentabbreviations{\indent 
 \begin{acronym}[ANOVA] % put the longest/widest acronym between the square brackets; it defines the left hand column width
 \acro{ANOVA}{ANalysis Of VAriance}
 \acro{CCA}{Canonical Correlation Analysis}
 \acro{DMN}{Default Mode Network}
 \acro{ERP}{Event-Related Potential}
 \acro{fMRI}{Functional Magnetic Resonance Imaging} 
 \acro{PCA}{Principle Component Analysis}
 \acro{MDES}{Mulit-Dimension Experience Sampling}
 \acro{SCCA}{Sparse Canonical Correlation Analysis}
\end{acronym}
}

% abstract text
\def\contentabstract{

The functional outcome of on-going thought is heterogeneous. However, the underlying mechanism of on-going thought and the reason of its heterogeneity remains a mystery. Neural hierarchy in unconstrained processes reveals a network can have multiple configuration. DMN emerges in demanding task and internal content representation each with an unique whole brain configuration. On-going thought as an unconstrained process would be heterogeneous due to the the wide variation of neural hierarchical organization. The current thesis aims to uncover the neuro-cognitive mechanism of on-going thought and establish the basic component processes. The current result suggests on-going thought is a family of heterogeneous member with overlapping similarities at individual level. The representational aspect of on-going thought is supported by the dissociation btween the DMN and the sensory system. Dysfunctions in executive control is contributed by the reduced connectivity in the MD network. However, the structure of the component processes remains unclear. Future studies on on-going thought or unconstrained process should consider a multivariate whole-brain approach to consider the hierarchy in cortical organization rather than individual regions. Investigation of transient on-going thought will be the future direction of research for a deeper understanding of the structure behind the component processes.
}


% Shortcuts and assorted macros
\def\mparetasquared{\eta_{p}^{2}}
\def\paretasquared{$\eta_{p}^{2}$}

\makeatletter
\newcommand*{\rom}[1]{\expandafter\@slowromancap\romannumeral #1@}
\makeatother


% Math operators and function names
\DeclareMathOperator{\avg}{avg}
\renewcommand{\vec}[1]{\mbox{$ \overrightarrow{ #1 } $}}