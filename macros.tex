% Line spacing tweaks
% UoY thesis demands 1.5x but sometimes we need to drop to 1.0x in tables, algorithms, and other awkward elements.
% Call \linespacesmall before the table, then \linespacenormal after the table.
\def\linespacesmall{
\renewcommand{\baselinestretch}{1.0}
}
\def\linespacenormal{
\renewcommand{\baselinestretch}{1.5}
}

% declaration text
% full references in text are all hand-edited. Package bibentry and hyperref have an unsolved conflict. To enable hyperref, I chose not to use bibentry.
\def\contentdeclaration{

I, Hao-Ting Wang, declare that this thesis is a presentation of original work and I am the sole author. I undertook the research at University of York during \yearstartedtext{} -- \yearendedtext{}, under the joint supervision of Professor Jonathan Smallwood and Professor Elizabeth Jefferies. This work has not previously been presented for an award at this, or any other, University. All sources are acknowledged as References.

Some parts of this thesis have been published in peer-reviewed journals or is currently under preparation for publication. Author contributions are noted at the start of each chapter.  

\begin{itemize}
    \item \cref{ch:methods}: Wang, H.-T., Smallwood, J., \& Bzdok, D. (2018). Finding the needle in high dimensions: A tutorial on CCA in biomedicine. Manuscript preparing for publication.	
	\item \cref{ch:study1}: Wang, H.-T., Poerio, G. L., Murphy, C. E., Bzdok, D., Jefferies, E., \& Smallwood, J.(2018). Dimensions of Experience: Exploring the Heterogeneity of the Wandering Mind. \textit{Psychological Science}, \textit{29} (1), 56--71. doi: \url{10.1177/0956797617728727}
	\item \cref{ch:study2}: Wang, H.-T., Bzdok, D., Margulies, D. S., Craddock, C., Milham, M., Jefferies, E., \& Smallwood, J.(2018). Patterns of thought: Population variation in the associations between large-scale network organisation and self-reported experiences at rest. \textit{NeuroImage}, \textit{176} (1), 518--527. doi: \url{10.1016/j.neuroimage.2018.04.064}
	
\end{itemize}

}

% acknowledgements text
\def\contentacknowledgement{

The most exciting chapter of my academic journey would not be possible without my supervisors Prof. Jonathan Smallwood and Prof. Elizabeth Jefferies. Jonny and Beth are the kindest people I have ever known. Especially, I cannot be more thankful for Jonny's trust in me when I was doubtful of my own ability. That offer really changed my life. Their help and support in both science and life have made this PhD a truly rewarding experience. From them, I learned to be a better scientist and also a good person. 

I express my gratitude to Prof. Dr. Dr. Danilo Bzdok for the guidance and discussions on the canonical correlation analysis project. Our discussions at various Brainhacks helped tremendously in the research methods of this PhD. 

I thank my thesis advisory panel, Dr. Tom Hartley, Dr. Aidan Horner, and Dr. Cade McCall for providing a friendly environment to discuss science and my skill development. 

All the research projects presented here will not be possible without the current and past members of semantics and thought lab. Their contributions lie not just in the research, and also moral supports on my personal life. Good colleagues like them are difficult to come by. 

Finally, thanks to all my family and friends for all their assistance. The special mentions go to Thomas Hardman and Rebecca Jones for making my life more fun in general. 

}

% acronyms and abbreviations text
\def\contentabbreviations{\indent 
 \begin{acronym}[ANOVA] % put the longest/widest acronym between the square brackets; it defines the left hand column width
 \acro{ANOVA}{ANalysis Of VAriance}
 \acro{CCA}{Canonical Correlation Analysis}
 \acro{DMN}{Default Mode Network}
 \acro{ERP}{Event-Related Potential}
 \acro{fMRI}{Functional Magnetic Resonance Imaging} 
 \acro{PCA}{Principle Component Analysis}
 \acro{MDES}{Mulit-Dimension Experience Sampling}
 \acro{SCCA}{Sparse Canonical Correlation Analysis}
\end{acronym}
}

% abstract text
\def\contentabstract{
Functional outcomes of ongoing thought show both costs and benefits. Yet, the reason for its heterogeneity remains unclear. The executive failure and representational accounts stemmed from different psychological research approaches to understand ongoing thought. The executive failure account examines why changes in ongoing thought happen, while the representational account seeks to explain how humans generate ongoing thought. The attentional system and the default mode network are the common neural processes of both theoretical accounts, but interacting in a contradicting manner. The two accounts can be seen as competing theories of ongoing thought. However, in the family resemblance view \cite{Seli2018}, the two theoretical accounts potentially serve as two component processes of one phenomenon. One possible solution to this conflict could be that under different global neural configurations, the two networks support different cognitive functions. The thesis sets out to present evidence supporting of the family resemblance view and to begin research on the ontology of the component processes in ongoing thought. Neural cognitive hierarchy is the potential explanation of the heterogeneity. The current thesis adopts sparse canonical correlation analysis to incorporate the neural and behavioural aspects of ongoing thought. The data suggests ongoing thought is a collective phenomenon with many types of experience driven by the connectivity patterns in the default mode network. Each type of experience associated with their unique functional outcomes and neural hierarchies at the whole-brain level. Cognitive flexibility and the balance of segregation and integration between the transmodal systems and the rest of the cortex determines the immersive details. The current findings suggested the importance of whole-brain neural hierarchies to ongoing thought. The confirmation of these trait level findings at a state level are necessary to gain more insights into the architecture of the component processes.
}


% Shortcuts and assorted macros
\def\mparetasquared{\eta_{p}^{2}}
\def\paretasquared{$\eta_{p}^{2}$}

\makeatletter
\newcommand*{\rom}[1]{\expandafter\@slowromancap\romannumeral #1@}
\makeatother


% Math operators and function names
\DeclareMathOperator{\avg}{avg}
\renewcommand{\vec}[1]{\mbox{$ \overrightarrow{ #1 } $}}