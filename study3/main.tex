\chapter{Inhibition of Prior Information Contributes to Internal Content Representation}
\chaptermark{Information inhibition and internal content representation}
\label{ch:study3}
%\setcounter{equation}{0}
% ====================================================================================================================
\section{Abstract}
Although the patterns of ongoing thought that make up our day to day lives are important, we know relatively little about how these experiences are constrained by an individuals' neurocognitive architecture. In the current study we used machine learning to identify stable patterns describing shared variance between performance on a battery of cognitive tasks in the laboratory, and intrinsic neural architecture observed at rest. Next we explored whether these dimensions explained variance in measures of ongoing thought recorded in the laboratory. We identified five neurocognitive dimensions characterised at the cognitive level as creative association, fluid intelligence, temporally specified cognition, and, separate dimensions of episodic memory linked to visual and verbal codes of representation. Variation in \textit{temporally specified cognition}---the ability to inhibit a prior mental set during task switching---predicted substantial variance in ongoing experience recorded in the lab, accounting for reduced variance in two principle dimensions identified by principal component analysis---less off-task thought and less immersive detail. In neural terms temporally specified cognition was characterised by patterns of high within-network limbic connectivity coupled with relative isolation between this system and other regions of cortex. Together this analysis suggests that whether an individuals thoughts are a pristine representation of the moment, or an immersive experience generated via imagination, depends cognitively on the ability to inhibit prior mental sets, and neurally on the balance of segregation and integration between the limbic system and the rest of the cortex.

% ====================================================================================================================
\section{Introduction}
\label{study3:intro}
Human cognition is not always tethered to the events in the here now. Phenomena such as mind-wandering highlight that we can become immersed in experiences generated from memories, rather than information in the immediate environment \cite{SmallwoodSchooler2015}. Although understanding self-generated experiences may ultimately inform theoretical accounts of both normal cognition and disease states, we currently lack a comprehensive understanding of how these experiences are constrained by an individuals' neurocognitive architecture.

Contemporary studies have shown that ongoing thought shares important links with both brain and behaviour. For example, the occurrence of off-task thought can jeopardise the integrity of tasks depending on executive control \cite{McVayJOEP2009,MrazekJoEP2012}.
On the other hand, states of off-task thought and daydreaming can be associated with better performance on tasks of memory and creativity \cite{RubyPlos2013,Poerio2017,WangPsychScience2018,Baird2012}.
Neuroimaging studies have highlighted important roles for a number of large-scale brain networks \cite<see  meta-analysis from>{Fox2015,Stawarczyk2015}.
These networks include the default mode network \cite{MasonScience2007,Christoff2009},
the frontoparietal and attention networks \cite{WangNI2018,Golchert2017}
and the limbic system \cite{Ellamil2016,Smallwood2016,Golchert2017}.

Associations between patterns of ongoing thought with objective metrics defined from brain and behaviour, raise the possibility that these metrics of individual difference could be used to gain traction on the architecture that underlies patterns of ongoing thought. In the current study, a large cohort of participants (\(n = 197\)) performed a battery of neurocognitive tasks in the laboratory, and, on a separate session, we measured their intrinsic brain activity using resting-state functional magnetic resonance imaging (fMRI).  These individuals also provided descriptions of their experience while they performed a simple laboratory task across several days.

Using these data we build on our prior work that used sparse canonical correlation analysis to reveal neurocognitive dimensions that relate to patterns of ongoing experience \cite{WangPsychScience2018,WangNI2018}. In this study, we used sparse canonical correlation analysis to identify dimensions that linked brain organisation to behaviour and used these as explanatory variables in analyses predicting patterns of ongoing thought in the laboratory. This allows us to test the view that ongoing thought is an emergent property of an individuals' neural architecture \cite<i.e.>{Gratton2018}.


% ==========================================================================================================

\section{Method}
\label{study3:method}
\subsection{Participants}
\label{study3:method:a}
Two hundred and seven healthy participants were recruited from the University of York (132 females, 65 males; age range = 18–-31 years, \textit{M} = 20.21, \textit{SD} = 2.36).
This analysis included two data sets with some shared measurements and the same MRI protocol as \cref{ch:study1}.
Participants were right-handed native English speakers with normal or corrected-to-normal vision and no history of psychiatric or neurological illness. Participants underwent MRI scanning, completed a 1-hr online questionnaire. The first cohort is identical to the sample in \cref{ch:study1}. Participant attended three
(165 participants; 99 females, 66 males; age range = 18–-31 years, \textit{M} = 20.43, \textit{SD} = 2.63) 2-hr behavioural testing sessions to complete a battery of cognitive tasks.
The second cohort (42 participants; 33 females, 9 males; age range = 18–-23 years, \textit{M} = 19.79, \textit{SD} = 1.37) underwent two 2-hr behavioural testing sessions to complete a battery of cognitive tasks. The behavioural sessions took place within a week of the scan. Ten participants were excluded from the multivariate pattern analysis because they failed to complete all of the behavioural testing sessions. In total, 197 participants (126 females, 71 males; age range = 18–-31 years, \textit{M} = 20.11, \textit{SD} = 2.24) were included in the multivariate pattern analysis and the comparison with cognitive performance. Participants were rewarded with either a payment of \pounds 10 per hour or a commensurate amount of course credit. All participants provided written consent prior to the fMRI session and the first behavioural testing session. Approval for the study was obtained from the ethics committee of the University of York Department of Psychology and the University of York Neuroimaging Centre.

\subsection{MRI acquisition}
\label{study3:method:b}
The MRI acquisition protocol was identical to the study documented in \cref{ch:study1}. Please refer to \cref{study1:method:b} for details.

\subsection{Resting state data preprocessing}
\label{study3:method:c}

All preprocessing and denoising steps for the MRI data were carried out using the SPM software package (Version 12.0) and Conn functional connectivity toolbox (Version 17.f), based on the MATLAB platform (Version 17.a). The first three functional volumes were removed in order to achieve steady-state magnetisation. The remaining data were first corrected for motion using six degrees of freedom (x, y, z translations and rotations), and adjusted for differences in slice-time. Subsequently, the high-resolution structural images were co-registered to the mean functional image via rigid-body transformation, segmented into grey/white matter and cerebrospinal fluid probability maps and all functional volumes were spatially normalized to Montreal Neurological Institute (MNI) space using the segmented images and a priori templates. This indirect procedure utilises the unified segmentation–normalization framework, which combines tissue segmentation, bias correction, and spatial normalization in a single unified model. No smoothing was employed, complying with recent studies that report the negative influence of this procedure on the construction of connectivity matrices analysis.

Moreover, a growing body of literature indicates the potential influence of participant motion inside the scanner on the subsequent estimates of functional connectivity. To ensure that motion and other artefacts did not confound our data, we have employed an extensive motion-correction procedure and denoising steps, comparable to those reported in the literature. In addition to the removal of six realignment parameters and their second-order derivatives using the general linear model (GLM), a linear detrending term was applied as well as the CompCor method that removed five principal components of the signal from white matter and cerebrospinal fluid. Moreover, the volumes affected by motion were identified and scrubbed based on the conservative settings of motion greater than 0.5 mm and global signal changes larger than z = 3. Though recent reports suggest the ability of global signal regression to account for head motion, it is also known to introduce spurious anti-correlations and was thus not utilised in our analysis. Finally, a band-pass filter between 0.009 Hz and 0.08 Hz was employed in order to focus on low-frequency fluctuations.

\subsection{ROI-ROI functional connectivity.}
\label{study3:method:d}
We adopted a set of 57 regions based on the Yeo 7 networks. We split the networks into two hemispheres and extracted clusters. Two voxels are considered connected only if they are adjacent within the same x, y, or z-direction. This yielded 57 clusters from the Yeo 7 networks parcellation. The implementation of spatial clusters extraction was retrieved from python library Nilearn
\cite[ \url{http://nilearn.github.io/}, version 0.3.1]{Abraham2014}.
Fully-connected, undirected and weighted matrices of bivariate correlation coefficients (Pearson's r) were constructed for each participant using the average BOLD signal time series obtained from all the 57 ROIs described above. The off-diagonal of each correlation matrix contained 1596 unique region-region connection strengths (i.e., the upper or lower triangle of the network covariance matrix). This approach provided a measure of the connection strength of the whole brain for each participant. Finally, Fisher's r-to-z transformation was applied to each network covariance matrix.

\subsection{Behavioural data}
\label{study3:method:e}

\subsubsection{Cognitive tasks}
\label{study3:method:e:cog}
We selected 9 cognitive tasks that are common across the two cohorts. The selected tasks measures cognitive functions that have been examined in previous mind-wandering literature, using the same or similar tasks adopted by previous mind-wandering research, including executive control (digit span: \cite<from>{Wechsler1999}, task switching task: \citeA{Handy2015}), generation of information (unusual uses task: \citeA{MrazekJoEP2012}, verbal fluency task: \citeA{Adlam2010} and \citeA{Balota2008}), semantic memory (semantics relatedness judgement tasks and feature matching task, both developed by \citeA{Krieger-Redwood2012}), episodic memory (paired-associate task: \citeA{Cairney2016}, four mountains task: \citeA{Hartley2007}), and fluid intelligence (Raven Advanced Progressive Matrices; RAPM: \citeA{Baird2012}). Please refer to Appendix \ref{appendix:study1:subsection2} for the detailed descriptions and references of the tasks.

Thirteen cognitive scores were calculated from the selected tasks. Performance of the digit span task was represented as the average of digit span in the forward and backward recall conditions. The verbal fluency score is the contrast of the letter condition and the category condition (letter--category). Picture naming tasks, the four mountains tasks, RAMP were summarised with accuracy scores. The task switching tasks provided two scores (a) flexibility
\footnote{
In the original contrast `switch cost', smaller values indicates a better ability to switch away from the previous condition. For the ease of interpretation, we reversed the scores and re-named the contrast as flexibility.}
as the ability to switch from a different condition and (b) inhibition as the ability to suppress information from the previous trial. The calculation of the task switching contrast can be found in Appendix \ref{appendix:study1:subsection2:TS}. All the semantics related judgement tasks, feature mating task and the paired associate task were summarised using efficiency scores. The efficiency scores were calculated as reaction time divided by accuracy. A smaller score indicates better performance, thus the scores were reversed to ease the interpretation. For the semantics tasks, we calculated three reaction contrasts based on the semantics modules tested: (a) strength (strong--week), modality (picture--word), and (c) specificity (specific--general). All the scores were standardised as z-scores in the subsequent analysis. We defined outliers as scores greater than 3. The identified outliers were then imputed with medians of each variable. 

\subsubsection{Experience sampling}
\label{study3:method:e:mdes}
Multi-dimensional experience sampling (MDES) was used to collect thoughts during a 0-back/1-back working memory task. Please refer to \cref{ch:study1} \cref{study1:method:d} for the MDES data collection and \cref{tab:study1:1} for the detailed questions. The MDES questions were aggregated (a) across all conditions, (b) within the 0-back condition, and (c) within the 1-back condition. All the scores were standardised as z-scores in the subsequent analysis. We defined outliers as scores greater than 3. The identified outliers were then imputed with medians of each variable.


\subsubsection{Dimensions of ongoing thought}

For the purpose of analyses, the scores on the 13 MDES questions were entered into a PCA to describe the underlying structure of the participants' responses. Following prior studies \cite{Konishi2017,Medea2016,RubyPlos2013} we concatenated the responses of each participant in each task into a single matrix and employed a principal components reduction with varimax rotation (see the top panel of \cref{fig:study3:fig1}). We selected the number of components based on the elbow in the scree plot.

\subsection{Multivariate pattern analysis}
\label{study3:method:f}
\subsubsection{SCCA}
We performed a sparse canonical correlation analysis \cite<SCCA; see>{Hastie2015}
on the functional connectomes and the cognitive tasks, to yield latent components that reflect multivariate patterns across neural organisation and cognition \cite<For similar application, see>{WangPsychScience2018}.
SCCA maximised the linear correlation between the low-rank projections of two sets of multivariate data sets with a sparse model to regularise the decomposition solutions a process that helps maximise the interpretability of the results. The regularisation function of choice is the L1 penalty, which produces `sparse' coefficients, meaning that the canonical vectors (i.e., translating from full variables to a data matrix's low-rank components of variation) will contain a number of exactly zero elements. L1 regularisation conducted (a) feature selection (i.e., select only relevant components) and (b) model estimation (i.e., determine what combination of components best disentangles the neurocognitive relationship) in an identical process. This way we handle adverse behaviours of classical linear models in high-dimensional data. A reliable and robust open-source implementation of the SCCA method was retrieved as R package from CRAN
\cite<PMA, penalized multivariate analysis, version 1.0.9>{Witten2009}.
%(PMA, penalized multivariate analysis, version 1.0.9, Witten, Tibshirani, \& Hastie, 2009).
The amount of L1 penalty for the functional connectomes and cognitive task performance were chosen by cross-validation. The procedure is described below.

\subsubsection{Model Selection}
\label{study3:method:f:1}

The model selection process was conducted with two parts: the L1 penalty coefficient selection and component selection. For the L1 penalty coefficient selection, we performed a grid search combined with cross-validation (CV) to avoid over-fitting \cite{BzdokYeo2017}.
Of each penalty pair on the search grid, a 10-fold CV was performed to search for the best out-of-sample the rank-1 canonical correlation (see the middle panel of \cref{fig:study3:fig1}). We then decomposed the full dataset with the selected L1 penalty coefficients. The K-Fold CV was conducted by the implementation in Python library scikit-learn \cite<\url{http://scikit-learn.org/stable/}, version 0.18.2>{scikit-learn}.

Confound removal was performed on the functional connectomes and the cognitive scores as suggested by prior study \cite{Smith2015}.
The confound variables were sex, age, and head motion indicated by mean frame-wise displacement \cite{Jenkinson2002}.
The removal steps were performed on the training set in each CV fold. The z-scores of the confound variables were calculated, and also squared the three confound measures to account for potentially nonlinear effects of these confounds. The 6 resulting confounds were regressed out of both data matrices.
The implementation of the confound removal method \cite{Friston1994} was retrieved from Python library Nilearn \cite[ \url{http://nilearn.github.io/}, version 0.3.1]{Abraham2014}.

After finding the optimal hyper-parameters, 1000 permutation tests with family-wise error (FWE) correction was applied to access the component(s) that occur above chance (see the bottom panel of \cref{fig:study3:fig1}). We constructed a null distribution for each canonical component by holding the functional connectivity data in place and randomising the row order of self-report data. This permutation scheme broke the link of individual differences in the dataset, therefore testing the robustness of the components in the hypothetical population. By calculating the false-discovery rate in the null distribution, we can conclude the possibility of discovering our components by chance with the given penalty coefficients. Hypotheses that are accepted with a 5\% level of significance. In the current analyses, we adopt the permutation test with the FWE-corrected p-value by Smith and colleagues \citeyear{Smith2015}. All components were compared to the first sparse canonical correlation of the permuted sample. The low-rank components are more relevant than the rest, therefore we yield more conservative p-value by comparing to the first canonical correlation only.

\begin{figure}[H]
    \centering
    \includegraphics[width=1\textwidth]{study3/image/study3fig1.jpg}
    \caption{Analysis pipeline.}
    \label{fig:study3:fig1}
    \caption*{\footnotesize{
    Top: PCA on MDES (\cref{study3:method:e:mdes}).
    Middle: sparsity parameter selection (\cref{study3:method:f:1}). 
    Bottom: permutation test procedure (\cref{study3:method:f:1}).
    }} % eleborate
\end{figure}

\subsection{Group analysis}
\label{study3:method:g}

To determine how patterns of unconstrained neurocognitive activity related to performance on the self-report experience summarised in three different ways (see section \ref{study3:method:e:mdes}),
we conducted three independent statistical analysis on the identical subjects. A Type III multivariate multiple regression with Pillai's trace test was applied to the data. Each of the latent components describing the neurocognitive mechanism from the SCCA was the independent variables, and the 13 measures from MDES were the dependent variables. We hoped to described the neurocognitive components by the linear combination of the self-report questions collected via MDES. The p-values reported were based on Bonferroni correction. The analysis was conducted in R (version 3.3.1). The multivariate multiple regression was conducted in R (version 3.3.1) using function \texttt{Manova} in R package \texttt{car} (companion to applied regression, version 2.1–5).

% ==========================================================================================================
\section{Results}
\label{study3:results}

\begin{wrapfigure}{o}{0.5\textwidth}

    \centering
    \includegraphics[width=0.48\textwidth]{study3/image/study3pca.png}
    \caption{Dimensions of ongoing though.}
    \caption*{The result from the PCA is presented as a heatmap. The colour bar represents the value of the principal component loading.}
    \label{fig:study3:figPCA}

\end{wrapfigure}

\subsection{Dimensions of ongoing thought}
Experience sampling probes revealed four unique dimensions of ongoing thought during the 0-back/1-back task. PCA of the 13 experience sampling questions resulted in four principle components of thought presented in \cref{fig:study3:figPCA}. Consistent with our prior study \cite{Poerio2017}, the components were characterised as: detailed thought, off-task thought, modality of thought, and emotional thought.


\subsection{Neurocognitive component selection}

Sparse Canonical Correlation Analysis (SCCA) was used to determine connectome-wide dimensions that describe common variance shared by descriptions of brain and experience. This took as input individual scores for the connections between each of the regions extracted from Yeo's 7 networks parcellation and the 13 cognitive task scores.

\begin{wrapfigure}{o}{0.5\textwidth}
    \centering
    \includegraphics[width=0.48\textwidth]{study3/image/study3fig2.png}
    \caption{Grid search result.}
    \caption*{This heatmap represents the rank-1 canonical correlations of each sparsity coefficient pairs determined by the CV. The red square indicates the best result.}
    \label{fig:study3:fig2}
\end{wrapfigure}

We applied SCCA with 10-fold CV and permutation tests as the model selection strategy. We obtained penalty levels of 0.6 on both the functional connectivity and cognitive tasks indicated the best out-of-sample prediction on our data through the grid search (Figure \ref{fig:study3:fig2}), obtaining 0.70 on the rank-1 canonical correlation. Five significant canonical components were identified through FWE-corrected p-value through permutation tests. The p-values of the 5 components are 0.028, 0.042, 0.041, 0.012, 0.033. The canonical correlations of the 5 significant latent components were 0.68, 0.68, 0.68, 0.70, and 0.68. Since the sparsity turns CCA into a convex optimisation problem, the modes didn't come out in the descending order of their canonical correlations.

\subsection{Determining constituent category of cognitive functions}

The latent components yielded by the best model are presented in \cref{fig:study3:fig3}. Using SCCA we identified five neurocognitive dimensions characterised at the cognitive level as the creative association (Component 1), fluid intelligence (Component 9), temporally specified cognition (Component 7), and, separate dimensions of episodic memory linked to visual (Component 8) and verbal (Component 12) codes of representation. For the ease of presentation and interpretation, we summarised the components as network-network connectivity instead of 57-by-57 connectivity matrices. The heat maps describe the network-to-network correlations and the cognitive tasks.

\begin{figure}[H]
    \centering
    \includegraphics[width=1\textwidth]{study3/image/study3fig3.png}
    \caption{Significant components from SCCA.}
    \caption*{The colour in the heatmap indicates the value of the canonical coefficients of each components. VIS: visual network, S-M: somatomotor network, VAN: ventral attention network, DAN: dorsal attention network, LIM: limbic network, FPN: frontoparietal network, DMN: default mode network.}
    \label{fig:study3:fig3}
\end{figure}

% explain the significant components
Component 1 emphasis semantic control with the picture-naming task and the unusual uses task, along with intelligence in the cognition component. The negative coefficient in semantic strength indicated the ability to detect weaker semantics relationships. The functional connectivity pattern shows a general dissociation among all networks, except between the unimodal systems. Network segregation was especially pronounced between the sensorimotor network and the limbic system, and a general dissociation between the unimodal sensory system and the attention and transmodal regions. Component 1 demonstrated semantic control ability to generate mental representation and semantic associations. 

Component 7 reflects better performance on task switching tasks, both in terms of a reduced switch cost, and the ability to suppress prior mental sets. The performance was better for both visual semantic decisions and specific semantic imagery. In addition, participants performed better on the four mountains task and had a larger digit span. This pattern of performance was linked to the better performance of tasks that requires mental representations to be controlled across time (task switching, four mountains task and digit span) particularly with a particular emphasis on visual processing. The connectivity pattern shows strong connectivity between sensory systems. In addition, the visual system showed reduced connectivity between the attention systems, while the limbic system was less correlated with all systems other than the FPN. Within network connectivity was strong for the limbic, frontoparietal and default mode networks. Overall, this component demonstrates the ability to control representations over time and is linked to integrity within limbic and transmodal systems and separation between visual and attention systems.

Component 8 pits executive tasks against verbal episodic memory systems since it is linked to better paired-associate memory, verbal semantic memory and feature matching and worse task switching, digit span and verbal fluency. The functional connectivity pattern shows a general pattern of reduced network connectivity.  Exceptions to this include the unimodal networks – the visual network shows stronger connections to attention and default mode networks, while the sensorimotor network shows stronger coupling to the ventral attention and limbic networks. Within network connectivity is higher in the ventral attention, limbic and default mode networks. 

Component 9 is composed of tasks that rely on controlled processing (fluid intelligence, task switching, fluency and controlled semantic retrieval). It is also linked to worse picture naming. In connectivity terms, the default mode network shows stronger connectivity with unimodal and attention networks, and the dorsal attention network is linked to stronger connectivity with the visual network. Within network connectivity is low within the visual, dorsal attention and default mode networks and high in the ventral attention network. 

Component 12 highlights the between episodic memory (better paired-associate memory) and worse visual semantic memory (worse verbal semantics and poor figure matching). Fluency was better when organised alphabetically rather than by categories. The default mode network and the visual system showed a general coupling pattern with the other networks, while a strong dissociation of the limbic with the attention systems and the sensorimotor system. The component demonstrates a strong ability to retain and recall information that does not benefit from the semantic organisation.

\subsection{The relationship between neurocognitive components and self-reports on thoughts}
\label{study3:results:group}
% multivariate main effect

Three regression models were performed to evaluate the relationship between neurocognitive components and self-reports on thoughts: average scores of all thought probes thought probes in 0-back and 1-back conditions (\cref{fig:study3:fig4}). 

We first examined the relations of the neurocognitive components and the overall thought reports. In the model of all thought probes, we got multivariate main effect in component 7 (\textit{F}(13, 160) = 2.239, \textit{p} = .010, \paretasquared = .154)
and 12 (\textit{F}(13, 160) = 1.946, \textit{p} = .029, \paretasquared = .137). There was only one significant univeriate effect after Bonferroni correction, which is the negative correlation on self-question under component 7 
(\textit{b} = −0.25, 95\% CI = [−0.40, −0.10], \textit{t}(172) = -3.369, \textit{p} = .005). The results revealed two types of thought patterns. Component 7 focuses on information maintenance in cognitive tasks and the integration within the separated transmodal systems. The related thought patterns shows a low tendency in reporting personal issues. Although there were no significant contributing univariate pattern in component 12, we see a trend of deliberate, focused thought pattern with low imagery information related to retrieval of semantically novel associations.

Two models examined the average scores of thought probes during the 0-back and 1-back condition separately.  The aim is to uncover potential differences in thought reports under the two conditions. In the 0-back condition, only component 7
(\textit{F}(13, 160) = 1.924, \textit{p} = .031, \paretasquared = .135)
showed the main effect in multivaraite level. There was only one significant univeriate effect after Bonferroni correction, which is the self-question under the model for component 7 
(\textit{b} = −0.23, 95\% CI = [−0.37, −0.79], \textit{t}(172) = -3.03, \textit{p} = .017).
In the 0-back condition, the participants perform a visual matching task. The focused state during the 0-back condition is associated with more cognitive mechanism that sustains ongoing thought. 

The 1-back condition model showed significant results in
component 7
(\textit{F}(13, 160) = 2.192, \textit{p} = .012, \paretasquared = .151)
and component 12
(\textit{F}(13, 160) = 2.312, \textit{p} = .008, \paretasquared = .158). There was only significant univeriate effect in the model for component 7 after Bonferroni correction.
The significant variable is the self question
(\textit{b} = −0.26, 95\% CI = [−0.40, −0.11], \textit{t}(172) = -3.51, \textit{p} = .003)
and the habit question
(\textit{b} = −0.22, 95\% CI = [−0.37, −0.75], \textit{t}(172) = -2.97, \textit{p} = .020).
The 1-back condition required participants to maintain meaningless associations of the two shapes presented on the screen.

\begin{figure}[H]
    \centering
    \includegraphics[width=0.9\textwidth]{study3/image/study3fig4.png}
    \caption{Group level analysis on neurocognitive components and  self-report on thoughts.}
    \caption*{
    \footnotesize{The codes next to the component number indicate the significant level of the mutivariate results, and those in the coloured cells are for the univariate results. The colour represent the univariate \textit{b} value. P-value significant codes: `***': \(<0.001\);  `**': \; `*': \(<0.05\); `.': \(<0.01\)}
    }
    \label{fig:study3:fig4}
\end{figure}


\subsection{The relationship between neurocognitive components and ongoing thought patterns}

Pearson's correlations were calculated to explore the relationships between the neurocognitive components and the dimensions of ongoing thought. The temporally specific cognition component (Component 7) is negatively correlated to details and off-task components (\cref{fig:study3:figcorr}). 


\begin{figure}[H]
    \centering
    \includegraphics[width=0.6\textwidth]{study3/image/pca_cca_corr.png}
    \caption{The relationships between the neurocognitive components and the ongoing thought.}
    \caption*{This heatmap represents correlation between the principal components and the univariate level predictions in \cref{study3:results:group}.}
    \label{fig:study3:figcorr}
\end{figure}

% ==========================================================================================================
\section{Discussion}
\label{study3:discussion}

We set out to identify patterns that described the association between different aspects of cognition and the intrinsic organisation of the cortex and to explore whether these accounted for variations in patterns of ongoing thought. Using SCCA we identified five neurocognitive dimensions characterised at the cognitive level as the creative association (Component 1), fluid intelligence (Component 9), temporally specified cognition (Component 7), and, separate dimensions of episodic memory linked to visual (Component 8) and verbal (Component 12) codes of representation. In our subsequent analysis, we identified that variation in temporally specified cognition was associated with substantial variance in patterns of ongoing thought recorded in the laboratory. In particular, we found that this neurocognitive dimension was associated with variation in both the task relatedness of cognition and its level of immersive details. In the discussion, we consider the implications of these data for theoretical accounts of ongoing thought.

In neural terms, our CCA analysis suggests that the relative degree of integration/segregation of the limbic system is at the core of whether an individuals experience is a pristine reflection of their current external goal, or instead, they are immersed in thoughts generated through imagination. We found that individuals who maintained attention on the task in hand, tended to show a pattern of brain activity dominated by whether the limbic system was coupled to itself, but decoupled from other cortical areas, while individuals who reported off-task experiences with immersive qualities showed the reverse pattern (low within network coupling, and high between network coupling for the limbic system). These data add to a growing body of evidence that highlights limbic regions as critical for patterns of spontaneous thought. For example, lesions to the hippocampus are associated with reductions in off-task thinking \cite{McCormick2018} and episodic future thinking as part of a task \cite{Maguire2011,Race2011}. Likewise, semantic dementia, which targets the lateral and medial aspects of the temporal pole, reduced the capacity to imagine the future \cite{Irish2012,Viard2014}. Furthermore, hippocampus activity has been shown to be important for spontaneous thought during the occurrence of spontaneous thought \cite{Ellamil2016}, while its connectivity with regions of the default mode network is important for both episodic features of spontaneous experience \cite{Karapanagiotidis2017} as well as its immersive features \cite{Smallwood2016}. Based on our results, the contribution of limbic structures to spontaneous experience may depend on their coupling with other regions, allowing these hub regions to integrate information from across the cortex to create a mental scene \cite{Hassabis2009}. This account is broadly consistent with views of limbic structures, such as the hippocampus \cite{Moscovitch2016} and the anterior temporal lobe \cite{Lambon-Ralph2017} which are both thought to share a hub and spoke architecture in which their contribution to cognition arise from their capacity to integrate information from across the cortical mantle.

In our analysis, participants with whom the limbic system was relatively isolated within the cortical mantle, performed well in a task switching context that required them to suppress representations of a previous task set. Prior uses of this task paradigm have documented that this ability is linked to the tendency to ruminate. \citeA{Whitmer2007} found that individuals who were high on trait rumination were better than non-ruminators when switching back to a prior task. This pattern is broadly consistent with our data which shows that people who show the smallest cost from inhibiting a prior mental set, were more likely to reports patterns of ongoing thought that were characterised by immersive experiences characterised by self-relevant, social and episodic content. Building on our study a promising area for future study would be to examine how limbic connectivity supports the recurrence of negative self-relevant experiences that are thought to be important in rumination \cite{Kleckner2017,Peters2016,Cooney2010}. 

Our data suggest that the default mode network shows a similar, albeit less pronounced pattern, to the limbic system. Given evidence of a role for the DMN in both immersive experiences in task states \cite{Sormaz2018,Richter2016} as well as in off-task states \cite{MasonScience2007,Christoff2009,StawarczykPlos2011}. It is possible that these two networks work in tandem when cognition is focused on self-relevant information with the limbic systems providing the episodic and conceptual content, and the default mode network allowing this content to be represented at a relatively abstract level. This interpretation is consistent with the observation that the default mode network is spatially located at the top of a hierarchy and most distant from unimodal inputs, while limbic regions occupy an intermediate position \cite{Margulies2016}. 

Finally, it is worth considering a number of limitations of this study. First, we did not measure patterns of ongoing thought while individuals performed the battery of cognitive tasks. It is, therefore, possible that part of the shared variance that our analyses capture emerges because of the patterns of ongoing thought occur during the cognitive tasks \cite<see>[for evidence of a similar point in the context of executive control or intelligence tasks]{MrazekJoEP2012}. However, this interpretation of our data is unlikely since individuals who were off-task tended to perform better on the tasks when they returned to a mental set that had recently been active. Second, our neural data was only measured on a single occasion, raising the possibility that this measure of brain function reflects a state rather than a trait. While this remains a possibility, recent studies have shown that individual patterns of functional connectivity remain relatively consistent across tasks and time \cite{Gratton2018}. Nonetheless, future studies could benefit from measuring an individuals architecture across multiple points to provide a more robust indication of its trait like features.

